% Options for packages loaded elsewhere
\PassOptionsToPackage{unicode}{hyperref}
\PassOptionsToPackage{hyphens}{url}
\PassOptionsToPackage{dvipsnames,svgnames*,x11names*}{xcolor}
%
\documentclass[
]{article}
\usepackage{lmodern}
\usepackage{amssymb,amsmath}
\usepackage{ifxetex,ifluatex}
\ifnum 0\ifxetex 1\fi\ifluatex 1\fi=0 % if pdftex
  \usepackage[T1]{fontenc}
  \usepackage[utf8]{inputenc}
  \usepackage{textcomp} % provide euro and other symbols
\else % if luatex or xetex
  \usepackage{unicode-math}
  \defaultfontfeatures{Scale=MatchLowercase}
  \defaultfontfeatures[\rmfamily]{Ligatures=TeX,Scale=1}
\fi
% Use upquote if available, for straight quotes in verbatim environments
\IfFileExists{upquote.sty}{\usepackage{upquote}}{}
\IfFileExists{microtype.sty}{% use microtype if available
  \usepackage[]{microtype}
  \UseMicrotypeSet[protrusion]{basicmath} % disable protrusion for tt fonts
}{}
\makeatletter
\@ifundefined{KOMAClassName}{% if non-KOMA class
  \IfFileExists{parskip.sty}{%
    \usepackage{parskip}
  }{% else
    \setlength{\parindent}{0pt}
    \setlength{\parskip}{6pt plus 2pt minus 1pt}}
}{% if KOMA class
  \KOMAoptions{parskip=half}}
\makeatother
\usepackage{xcolor}
\IfFileExists{xurl.sty}{\usepackage{xurl}}{} % add URL line breaks if available
\IfFileExists{bookmark.sty}{\usepackage{bookmark}}{\usepackage{hyperref}}
\hypersetup{
  pdftitle={Examen 3A},
  pdfauthor={Cruz Martínez Ricardo y Vasquez Arriaga Jorge},
  colorlinks=true,
  linkcolor=Maroon,
  filecolor=Maroon,
  citecolor=Blue,
  urlcolor=blue,
  pdfcreator={LaTeX via pandoc}}
\urlstyle{same} % disable monospaced font for URLs
\usepackage[margin=1in]{geometry}
\usepackage{color}
\usepackage{fancyvrb}
\newcommand{\VerbBar}{|}
\newcommand{\VERB}{\Verb[commandchars=\\\{\}]}
\DefineVerbatimEnvironment{Highlighting}{Verbatim}{commandchars=\\\{\}}
% Add ',fontsize=\small' for more characters per line
\usepackage{framed}
\definecolor{shadecolor}{RGB}{248,248,248}
\newenvironment{Shaded}{\begin{snugshade}}{\end{snugshade}}
\newcommand{\AlertTok}[1]{\textcolor[rgb]{0.94,0.16,0.16}{#1}}
\newcommand{\AnnotationTok}[1]{\textcolor[rgb]{0.56,0.35,0.01}{\textbf{\textit{#1}}}}
\newcommand{\AttributeTok}[1]{\textcolor[rgb]{0.77,0.63,0.00}{#1}}
\newcommand{\BaseNTok}[1]{\textcolor[rgb]{0.00,0.00,0.81}{#1}}
\newcommand{\BuiltInTok}[1]{#1}
\newcommand{\CharTok}[1]{\textcolor[rgb]{0.31,0.60,0.02}{#1}}
\newcommand{\CommentTok}[1]{\textcolor[rgb]{0.56,0.35,0.01}{\textit{#1}}}
\newcommand{\CommentVarTok}[1]{\textcolor[rgb]{0.56,0.35,0.01}{\textbf{\textit{#1}}}}
\newcommand{\ConstantTok}[1]{\textcolor[rgb]{0.00,0.00,0.00}{#1}}
\newcommand{\ControlFlowTok}[1]{\textcolor[rgb]{0.13,0.29,0.53}{\textbf{#1}}}
\newcommand{\DataTypeTok}[1]{\textcolor[rgb]{0.13,0.29,0.53}{#1}}
\newcommand{\DecValTok}[1]{\textcolor[rgb]{0.00,0.00,0.81}{#1}}
\newcommand{\DocumentationTok}[1]{\textcolor[rgb]{0.56,0.35,0.01}{\textbf{\textit{#1}}}}
\newcommand{\ErrorTok}[1]{\textcolor[rgb]{0.64,0.00,0.00}{\textbf{#1}}}
\newcommand{\ExtensionTok}[1]{#1}
\newcommand{\FloatTok}[1]{\textcolor[rgb]{0.00,0.00,0.81}{#1}}
\newcommand{\FunctionTok}[1]{\textcolor[rgb]{0.00,0.00,0.00}{#1}}
\newcommand{\ImportTok}[1]{#1}
\newcommand{\InformationTok}[1]{\textcolor[rgb]{0.56,0.35,0.01}{\textbf{\textit{#1}}}}
\newcommand{\KeywordTok}[1]{\textcolor[rgb]{0.13,0.29,0.53}{\textbf{#1}}}
\newcommand{\NormalTok}[1]{#1}
\newcommand{\OperatorTok}[1]{\textcolor[rgb]{0.81,0.36,0.00}{\textbf{#1}}}
\newcommand{\OtherTok}[1]{\textcolor[rgb]{0.56,0.35,0.01}{#1}}
\newcommand{\PreprocessorTok}[1]{\textcolor[rgb]{0.56,0.35,0.01}{\textit{#1}}}
\newcommand{\RegionMarkerTok}[1]{#1}
\newcommand{\SpecialCharTok}[1]{\textcolor[rgb]{0.00,0.00,0.00}{#1}}
\newcommand{\SpecialStringTok}[1]{\textcolor[rgb]{0.31,0.60,0.02}{#1}}
\newcommand{\StringTok}[1]{\textcolor[rgb]{0.31,0.60,0.02}{#1}}
\newcommand{\VariableTok}[1]{\textcolor[rgb]{0.00,0.00,0.00}{#1}}
\newcommand{\VerbatimStringTok}[1]{\textcolor[rgb]{0.31,0.60,0.02}{#1}}
\newcommand{\WarningTok}[1]{\textcolor[rgb]{0.56,0.35,0.01}{\textbf{\textit{#1}}}}
\usepackage{graphicx,grffile}
\makeatletter
\def\maxwidth{\ifdim\Gin@nat@width>\linewidth\linewidth\else\Gin@nat@width\fi}
\def\maxheight{\ifdim\Gin@nat@height>\textheight\textheight\else\Gin@nat@height\fi}
\makeatother
% Scale images if necessary, so that they will not overflow the page
% margins by default, and it is still possible to overwrite the defaults
% using explicit options in \includegraphics[width, height, ...]{}
\setkeys{Gin}{width=\maxwidth,height=\maxheight,keepaspectratio}
% Set default figure placement to htbp
\makeatletter
\def\fps@figure{htbp}
\makeatother
\setlength{\emergencystretch}{3em} % prevent overfull lines
\providecommand{\tightlist}{%
  \setlength{\itemsep}{0pt}\setlength{\parskip}{0pt}}
\setcounter{secnumdepth}{-\maxdimen} % remove section numbering
\usepackage[spanish]{babel}
\usepackage[utf8]{inputenc}
\decimalpoint
\usepackage{booktabs}
\usepackage{longtable}
\usepackage{array}
\usepackage{multirow}
\usepackage{wrapfig}
\usepackage{float}
\usepackage{colortbl}
\usepackage{pdflscape}
\usepackage{tabu}
\usepackage{threeparttable}
\usepackage{threeparttablex}
\usepackage[normalem]{ulem}
\usepackage{makecell}
\usepackage{xcolor}
\usepackage{fancyhdr}
\usepackage{lastpage}

\title{Examen 3A}
\author{Cruz Martínez Ricardo y Vasquez Arriaga Jorge}
\date{}

\begin{document}
\maketitle

\hypertarget{muestreo-bietuxe1pico}{%
\subsection{1 Muestreo bietápico}\label{muestreo-bietuxe1pico}}

Un estudiante de posgrado tiene una colección de 400 libros en su
librero. Para estimar el número total de palabras en su colección de
libros, selecciona una muestra \(s_I\) de dos libros usando un muestreo
aleatorio simple sin reemplazo. Para los dos libros seleccionados tiene
el número de páginas de cada uno. Posteriormente, en cada libro
seleccionado selecciona dos páginas usando un muestreo aleatorio simple
sin reemplazo. Una vez seleccionada la página registra de cada una el
número de palabras que contiene. La información muestral es como sigue.

\begin{table}[H]
  \centering
  \caption{Muestras seleccionadas en cada etapa y valores observados}
    \begin{tabular}{cccc}\\
    Libro  & Número &Páginas & Número de Palabras   \\
    seleccionado ($s_I$) &de páginas ($N_i$)  & seleccionadas ($s_i$) & en la página ($y_k$) \\ \hline
    195   & 300   & 61    &  23    \\
         &       & 212   &  25    \vspace{.2cm}\\
    288   & 200   & 99    & 20    \\
          &       & 111   & 20   \\ \hline
   \end{tabular}
\end{table}

Aquí las páginas conforman la población de interés y los libros son las
upm.

\textbf{i)} Calcule las probabilidades de inclusión de primer y segundo
orden correspondientes a las páginas seleccionadas.

Por el \textbf{Resultado 5.4} de las notas, tenemos que las
probabilidades de primer orden, cosiderando el muestreo bietápico están
dadas por \[\pi_k = \pi_{k|i}\pi_{Ii},\] donde \(\pi_{Ii}\) representa
lal probabilidad de inclusión de la \(upm\) \(i\) (en este caso, serían
los libros) y \(\pi_{k|i}\) la probabilidad de inclusión del elemento
\(k\) cuando se realiza la selección en la \(upm\) \(i\) (sería la
probabilidad de escoger una página del libro \(i\)) y dado que en la
primera y segunda `etapa' se usó un muestreo aleatorio simple sin
reemplazo, se tiene que las probabilidades de inclusión correspondientes
a las páginas seleccionadas son:
\[\pi_{61} = \frac{2}{400} \cdot \frac{2}{300} = \frac{1}{30000} = \pi_{212}\]
y
\[\pi_{99}  = \frac{2}{400} \cdot \frac{2}{200} = \frac{1}{20000} = \pi_{111},\]
pues \[\pi_{I_{195}} = \frac{2}{400} = \pi_{I_{288}},\] además
\[\pi_{61|195} = \frac{2}{300} = \pi_{212|195} \quad y \quad \pi_{99|288} = \frac{2}{200} = \pi_{111|288}.\]
Por otra parte, sabemos que las probabilidades de incluisión de segundo
orden están dadas por

\[ \pi_{kl}=\begin{cases} 
      \pi_{Ii}\pi_{kl|i} & \text{si } i=j \\
      \pi_{Iij}\pi_{k|i}\pi_{l|j} & \text{si } i\neq j  
   \end{cases}
\]

donde \(\pi_{Iij}\) representa la probabilidad de inclusión de segundo
orden para las \(upm\) \(i\) y \(j\), \(i\neq j\) y \(\pi_{kl|i}\) la
probablidad de inclusión de segundo orden de los elementos \(k\) y \(l\)
cuando se realiza la selección en la \(upm\) \(i\), entonces, se tiene
que
\[\pi_{61 \& 212} = \pi_{I_{195}}\pi_{61\&212 | 195} = \frac{2}{400}\cdot\frac{2}{300(299)} = \frac{1}{8970000}\]
\[\pi_{99\&111} = \pi_{I_{288}}\pi_{99\&111 | 288} = \frac{2}{400}\cdot \frac{2}{200(199)}= \frac{1}{3980000} \]
\[\pi_{61 \& 99} = \pi_{I_{195 \&288}}\pi_{61|195}\pi_{99|288} = \frac{2}{400(399)}\cdot\frac{2}{300}\cdot\frac{2}{200} = 8.354218881\times 10^{-10}\]
\[\pi_{61 \& 111} = \pi_{I_{195 \&288}}\pi_{61|195}\pi_{111|288} = \frac{2}{400(399)}\cdot\frac{2}{300}\cdot\frac{2}{200} = 8.354218881\times 10^{-10}\]
\[\pi_{212\&99} = \pi_{I_{195 \&288}} \pi_{212|195}\pi_{99|288} = \frac{2}{400(399)}\cdot\frac{2}{300} \cdot\frac{2}{200} = 8.354218881\times 10^{-10} \]
\[\pi_{212\&111} = \pi_{I_{195 \&288}} \pi_{212|195}\pi_{111|288} = \frac{2}{400(399)}\cdot\frac{2}{300}\cdot\frac{2}{200} = 8.354218881\times 10^{-10}\]
\textbf{ii)} Calcule el valor estimado del número total de palabras en
la colección de libros.

Por el \textbf{Resultado 5.5} el estimador HT para el total se ve como
\[\widehat{t}_{y\pi} = \sum_{i\in s_I}\frac{\widehat{t}_{i\pi}}{\pi_{Ii}}\]
con \(\widehat{t}_{i\pi} = \sum_{k\in s_i}\frac{y_k}{\pi_{k|i}}\), por
ende, tenemos que la estimación quedaría de la siguiente manera:

\begin{enumerate}
    \item Para $S_I = 195$, se tiene que
    $$ \widehat{t}_{195\pi} = \sum_{k \in \{61, 212\}}\frac{300}{2}y_k = 150(23 + 25) = 7200$$
    \item Para $S_I = 288$, se tiene
    $$ \widehat{t}_{288\pi} = \sum_{k \in \{99, 111\}}\frac{200}{2}y_k = 100(40) = 4000 $$
\end{enumerate}

Por último, nuestra estimación del total quedaría como:
\[ \widehat{t}_{y\pi} = \sum_{i \in \{195, 288\}}200 \widehat{t}_{i\pi} = 200(7200 + 4000) = 2240000\]
\textbf{iii)} Dé una estimación insesgada de la varianza del estimador
usado en ii).

Sabemos, por el \textbf{Resultado 5.6}, que un estimador insesgado de la
varianza del estimador del inciso anterior está dado por
\[\widehat{V}(\widehat{t}_{y\pi}) = \widehat{V}_{PSU} + \widehat{V}_{SSU} = \sum_{i\in s_I}\sum_{j\in s_I}\widehat{\Delta}_{Iij}\frac{\widehat{t}_{i\pi}}{\pi_{Ii}}\frac{\widehat{t}_{j\pi}}{\pi_{Ij}} + \sum_{i\in s_I}\frac{\widehat{V}_i}{\pi_{Ii}},\]

donde
\(\widehat{\Delta}_{Iij} = \frac{\pi_{Iij} - \pi_{Ii}\pi_{Ij}}{\pi_{Iij}}\)
y
\(\widehat{V}_i= \sum_{k\in s_i}\sum_{l\in s_i} \widehat{\Delta}_{kl|i}\frac{y_k}{\pi_{k|i}}\frac{y_l}{\pi_{l|i}}\),
con
\(\widehat{\Delta}_{kl|i} = \frac{\pi_{kl|i} - \pi_{k|i}\pi_{l|i}}{\pi_{kl|i}}\),
dado que en la primera y segunda etapa se usó un muestreo aleatorio
simple sin reemplazo, tenemos que
\[\sum_{i\in s_I}\sum_{j\in s_I}\widehat{\Delta}_{Iij}\frac{\widehat{t}_{i\pi}}{\pi_{Ii}}\frac{\widehat{t}_{j\pi}}{\pi_{Ij}}\]
es la expresión de la varianza del estimador HT para un muestreo
aleatorio simple sin reemplazo para la primera estapa, por lo que se
tiene que
\[\sum_{i\in s_I}\sum_{j\in s_I}\widehat{\Delta}_{Iij}\frac{\widehat{t}_{i\pi}}{\pi_{Ii}}\frac{\widehat{t}_{j\pi}}{\pi_{Ij}} = \frac{N^2}{n}\left( 1 - \frac{n}{N} \right)S^2_{\widehat{t_i}_\pi s_I}\]
con
\(S^2_{\widehat{t_i}_\pi s_I} = \frac{\sum_{k\in s_I}(\widehat{t}_{i\pi} - \overline{\widehat{t}_{\pi}})^2}{n-1}\),
\(\overline{\widehat{t}_{\pi}} = \frac{\sum_{i\in s_I}\widehat{t}_{i\pi}}{n}\),
\(N = 400\) y \(n = 2\).

\begin{Shaded}
\begin{Highlighting}[]
\NormalTok{s_I <-}\KeywordTok{c}\NormalTok{(}\DecValTok{195}\NormalTok{,}\DecValTok{195}\NormalTok{, }\DecValTok{288}\NormalTok{, }\DecValTok{288}\NormalTok{)}
\NormalTok{N_i <-}\KeywordTok{c}\NormalTok{(}\DecValTok{300}\NormalTok{,}\DecValTok{300}\NormalTok{, }\DecValTok{200}\NormalTok{, }\DecValTok{200}\NormalTok{)}
\NormalTok{s_i <-}\KeywordTok{c}\NormalTok{(}\DecValTok{61}\NormalTok{, }\DecValTok{212}\NormalTok{, }\DecValTok{99}\NormalTok{, }\DecValTok{111}\NormalTok{)}
\NormalTok{y_k <-}\KeywordTok{c}\NormalTok{(}\DecValTok{23}\NormalTok{, }\DecValTok{25}\NormalTok{, }\DecValTok{20}\NormalTok{,}\DecValTok{20}\NormalTok{)}

\NormalTok{Datos <-}\StringTok{ }\KeywordTok{data.frame}\NormalTok{(s_I, N_i, s_i, y_k)}

\NormalTok{t_i =}\StringTok{ }\KeywordTok{c}\NormalTok{(}\DecValTok{7200}\NormalTok{, }\DecValTok{4000}\NormalTok{)}
 
\NormalTok{PSU =}\StringTok{ }\KeywordTok{var}\NormalTok{(t_i)}\OperatorTok{*}\NormalTok{(}\DecValTok{400}\OperatorTok{^}\DecValTok{2} \OperatorTok{/}\StringTok{ }\DecValTok{2}\NormalTok{) }\OperatorTok{*}\StringTok{ }\NormalTok{(}\DecValTok{1} \OperatorTok{-}\StringTok{ }\DecValTok{2}\OperatorTok{/}\DecValTok{400}\NormalTok{)}
\NormalTok{PSU}
\end{Highlighting}
\end{Shaded}

\begin{verbatim}
## [1] 4.07552e+11
\end{verbatim}

De manera similar, tenemos que el segundo término del sumando, dado que
a cada \(V_i\) la podemos pensar como una suma de varianzas del
estimador HT aplicado a las páginas de los libros seleccionados, se
tiene que
\[\widehat{V}_i= \sum_{k\in s_i}\sum_{l\in s_i} \widehat{\Delta}_{kl|i}\frac{y_k}{\pi_{k|i}}\frac{y_l}{\pi_{l|i}} = \frac{N_i^2}{2}\left( 1 - \frac{2}{N_i} \right) S^2_{y s_i}\]
con
\(S^2_{y s_i} = \frac{\sum_{k\in s_i}(y_k - \overline{y}_{s_i})^2}{n-1}\)
y \(\overline{y}_{s_i} = \frac{\sum_{k\in s_i}y_k}{n}\)

\begin{Shaded}
\begin{Highlighting}[]
\KeywordTok{library}\NormalTok{(dplyr)}
\NormalTok{V_i =}\StringTok{ }\NormalTok{Datos }\OperatorTok\StringTok{ }\KeywordTok{group_by}\NormalTok{(s_I)}\OperatorTok\KeywordTok{summarise}\NormalTok{(}\DataTypeTok{V_i =}\KeywordTok{unique}\NormalTok{(N_i)}\OperatorTok{^}\DecValTok{2}\OperatorTok{/}\DecValTok{2}\OperatorTok{*}\NormalTok{(}\DecValTok{1}

     \OperatorTok{-}\StringTok{ }\DecValTok{2}\OperatorTok{/}\KeywordTok{unique}\NormalTok{(N_i))}\OperatorTok{*}\KeywordTok{var}\NormalTok{(y_k))}\OperatorTok\KeywordTok{ungroup}\NormalTok{()}

\NormalTok{SSU =}\StringTok{ }\KeywordTok{sum}\NormalTok{(V_i) }\OperatorTok{*}\StringTok{ }\NormalTok{(}\DecValTok{400} \OperatorTok{/}\StringTok{ }\DecValTok{2}\NormalTok{)}

\NormalTok{SSU}
\end{Highlighting}
\end{Shaded}

\begin{verbatim}
## [1] 17976600
\end{verbatim}

Finalmente, la estimación de la varianza estaría dada por

\begin{Shaded}
\begin{Highlighting}[]
\NormalTok{PSU }\OperatorTok{+}\StringTok{ }\NormalTok{SSU}
\end{Highlighting}
\end{Shaded}

\begin{verbatim}
## [1] 4.0757e+11
\end{verbatim}

\textbf{iv)} Dé una aproximación de la varianza del estimador usado en
ii) que sólo use las probabilidades de inclusión de primer orden o los
factores de expansión.

Sabemos, por la ecuación \((168)\) de las notas, que bajo los supuestos
de \(\pi_{Ii} = n_I p_i\) y \(m_I = n_I\) usando un muestreo con
reemplazo para seleccionar las upm, se tiene que una estimación de la
varianza es la siguiente
\[ \widehat{V}(\widehat{t}_{y\pi}) = \frac{n_I}{n_I - 1}\sum_{i = 1}^{n_I}\left(\frac{\widehat{t}_{i\pi}}{\pi_{Ii}} - \frac{\widehat{t}_{y\pi}}{n_I} \right)^2\]
en nuestro caso, tenemos que \(n_I = 2\),
\(\frac{\widehat{t}_{y\pi}}{2} = \frac{\frac{400}{2} \cdot (7200 + 4000)}{2} = \frac{2240000}{2} = 1120000\),
por ende, tenemos lo siguiente

\begin{Shaded}
\begin{Highlighting}[]
\NormalTok{t_ypi2 =}\StringTok{ }\DecValTok{1120000}
\NormalTok{estVar <-}\StringTok{ }\DecValTok{2} \OperatorTok{*}\StringTok{ }\KeywordTok{sum}\NormalTok{((}\DecValTok{400}\OperatorTok{/}\DecValTok{2} \OperatorTok{*}\StringTok{ }\NormalTok{t_i }\OperatorTok{-}\StringTok{ }\NormalTok{t_ypi2)}\OperatorTok{^}\DecValTok{2}\NormalTok{)}
\NormalTok{estVar}
\end{Highlighting}
\end{Shaded}

\begin{verbatim}
## [1] 4.096e+11
\end{verbatim}

\hypertarget{comparaciuxf3n-de-diseuxf1os-de-muestreo}{%
\subsection{2 Comparación de diseños de
muestreo}\label{comparaciuxf3n-de-diseuxf1os-de-muestreo}}

Suponga que se realizará una nueva elección de diputaciones a nivel
federal y le han encargado realizar el diseño de muestreo. Para esto
cuenta con la información de los resultados a nivel acta (casilla) de
los cómputos distritales de 2021
(\url{https://computos2021.ine.mx/base-de-datos}).

Por simplicidad suponga que se centrará en estudiar los diseños de
manera que elegirá el que sirva para realizar de la mejor forma la
estimación del porcentaje de votos a favor de una coalición integrada
por Morena, PT y P. Verde a nivel nacional.

En el denominador considerará sólo el total de votos válidos, es decir,
que no se considerarán los votos nulos para efectos del cálculo del
porcentaje.

Consideré los siguientes tres diseños a comparar:

\begin{enumerate}
\def\labelenumi{\roman{enumi}.}
\tightlist
\item
  Se seleccionan 1200 casillas de las 163,666 usando un muestreo
  aleatorio simple sin reemplazo.
\item
  Se considera una estratificación a partir de los 300 diferentes
  distritos electorales del país. El diseño de muestreo considerado en
  cada estrato corresponde a un muestreo aleatorio simple sin reemplazo
  de 4 casillas.
\item
  Se considera una estratificación considerando las cinco
  circunscripciones electorales de México. En cada estrato se usa un
  muestreo aleatorio simple sin reemplazo para seleccionar 8 distritos
  electorales y en cada distrito electoral seleccionado se usa un
  muestreo aleatorio simple para seleccionar 30 casillas.
\end{enumerate}

Realice la comparación de los tres diseños a partir de simulaciones. Es
decir, repita 1000 veces lo siguiente. Seleccione una muestra con cada
diseño y realice la estimación, en este caso deberá usar un estimador de
razón.

A partir de las 1000 estimaciones estime el ECM del estimador que se
obtendría para cada diseño y con estos resultados indique cuál diseño
parece ser el mejor.

Para resolver este problema optamos por resolver manualmente debido a
que con sampling(strata, stage) y survey, se tardaba demasiado tiempo.

\begin{Shaded}
\begin{Highlighting}[]
\CommentTok{# libreria para limpiar la base de datos}
\KeywordTok{library}\NormalTok{(tidyverse)}

\CommentTok{# leer el csv}
\KeywordTok{setwd}\NormalTok{(}\StringTok{"D:/Notas/Muestreo/Exámen/examen02/20210611_1000_CW_diputaciones"}\NormalTok{)}
\KeywordTok{options}\NormalTok{(}\DataTypeTok{digits=}\DecValTok{10}\NormalTok{)}
\KeywordTok{set.seed}\NormalTok{(}\DecValTok{123}\NormalTok{)}


\NormalTok{df <-}\StringTok{ }\NormalTok{data.table}\OperatorTok{::}\KeywordTok{fread}\NormalTok{(}\StringTok{"diputaciones - copia.csv"}\NormalTok{, }\DataTypeTok{sep=}\StringTok{"|"}\NormalTok{)}
\CommentTok{# para el INE 34. TOTAL_VOTOS_CALCULADOS -  Suma de los votos asentados en las}
\CommentTok{# actas para: los partidos políticos, combinaciones de estos, candidatos }
\CommentTok{# independientes, votos para candidaturas no registradas y votos nulos. }

\CommentTok{# como solo nos interesa el porcentaje sobre la gente que no registro voto nulo}
\CommentTok{# entonces los restamos}
\CommentTok{# y solo nos interesa las casillas donde si hubo votaciones}
\NormalTok{df <-}\StringTok{ }\KeywordTok{filter}\NormalTok{(df, TOTAL_VOTOS_CALCULADOS }\OperatorTok{!=}\StringTok{ }\DecValTok{0}\NormalTok{)}
\NormalTok{df <-}\StringTok{ }\NormalTok{df }\OperatorTok\KeywordTok{drop_na}\NormalTok{(TOTAL_VOTOS_CALCULADOS)}
\NormalTok{df <-}\StringTok{ }\NormalTok{df }\OperatorTok\KeywordTok{drop_na}\NormalTok{(}\StringTok{`}\DataTypeTok{VOTOS NULOS}\StringTok{`}\NormalTok{)}
\NormalTok{df}\OperatorTok{$}\NormalTok{totalvotos =}\StringTok{ }\NormalTok{df}\OperatorTok{$}\NormalTok{TOTAL_VOTOS_CALCULADOS }\OperatorTok{-}\StringTok{ }\NormalTok{df}\OperatorTok{$}\StringTok{`}\DataTypeTok{VOTOS NULOS}\StringTok{`}
\NormalTok{df <-}\StringTok{ }\KeywordTok{filter}\NormalTok{(df, totalvotos }\OperatorTok{!=}\StringTok{ }\DecValTok{0}\NormalTok{)}

\CommentTok{# el número total de casilla sobre el que vamos a trabajr}
\NormalTok{N <-}\StringTok{ }\KeywordTok{dim}\NormalTok{(df)[}\DecValTok{1}\NormalTok{]}
\CommentTok{# tamaño de muestra}
\NormalTok{n <-}\StringTok{ }\DecValTok{1200}

\CommentTok{# datos que nos interesan}
\NormalTok{midf <-}\StringTok{ }\NormalTok{df[, }\KeywordTok{c}\NormalTok{(}\StringTok{"MORENA"}\NormalTok{, }\StringTok{"PVEM"}\NormalTok{, }\StringTok{"PT"}\NormalTok{, }\StringTok{"PT-MORENA"}\NormalTok{,}
               \StringTok{"PVEM-PT"}\NormalTok{, }\StringTok{"PVEM-MORENA"}\NormalTok{, }\StringTok{"PVEM-PT-MORENA"}\NormalTok{, }\StringTok{"totalvotos"}\NormalTok{)]}
\CommentTok{# calculamos los votos a favor}
\NormalTok{midf}\OperatorTok{$}\NormalTok{votos.favor =}\StringTok{ }\KeywordTok{apply}\NormalTok{(midf,}\DecValTok{1}\NormalTok{, }\ControlFlowTok{function}\NormalTok{(x) (}\KeywordTok{sum}\NormalTok{(x)}\OperatorTok{-}\StringTok{ }\NormalTok{x[}\KeywordTok{length}\NormalTok{(x)]))}

\CommentTok{# resultado a estimar}
\NormalTok{prct.votos <-}\StringTok{ }\KeywordTok{sum}\NormalTok{(midf}\OperatorTok{$}\NormalTok{votos.favor)}\OperatorTok{/}\KeywordTok{sum}\NormalTok{(midf}\OperatorTok{$}\NormalTok{totalvotos)}\OperatorTok{*}\DecValTok{100}
\NormalTok{prct.votos}
\end{Highlighting}
\end{Shaded}

\begin{verbatim}
## [1] 44.28523292
\end{verbatim}

\begin{Shaded}
\begin{Highlighting}[]
\CommentTok{# pesos para el mas}
\NormalTok{midf}\OperatorTok{$}\NormalTok{wk.mas =}\StringTok{ }\NormalTok{N}\OperatorTok{/}\NormalTok{n}


\NormalTok{diseno.mas <-}\StringTok{ }\ControlFlowTok{function}\NormalTok{()\{}
  \CommentTok{#seleccionamos muestra}
\NormalTok{  seleccion_muestra <-}\StringTok{ }\KeywordTok{sample}\NormalTok{(}\DecValTok{1}\OperatorTok{:}\NormalTok{N ,}\DataTypeTok{size =} \DecValTok{1200}\NormalTok{)}
  
  \CommentTok{#estimamos con el estimador de razon de la media}
\NormalTok{  muestra <-}\StringTok{ }\NormalTok{midf[seleccion_muestra, ]}
  \KeywordTok{return}\NormalTok{(}\KeywordTok{sum}\NormalTok{(muestra}\OperatorTok{$}\NormalTok{votos.favor}\OperatorTok{*}\NormalTok{muestra}\OperatorTok{$}\NormalTok{wk.mas)}\OperatorTok{/}\KeywordTok{sum}\NormalTok{(muestra}\OperatorTok{$}\NormalTok{totalvotos}\OperatorTok{*}\NormalTok{muestra}\OperatorTok{$}\NormalTok{wk.mas))}
\NormalTok{\}}

\NormalTok{simulacion.mas <-}\StringTok{ }\KeywordTok{replicate}\NormalTok{(}\DecValTok{1000}\NormalTok{, }\KeywordTok{diseno.mas}\NormalTok{())}
\KeywordTok{mean}\NormalTok{(simulacion.mas)}\OperatorTok{*}\DecValTok{100}
\end{Highlighting}
\end{Shaded}

\begin{verbatim}
## [1] 44.30034631
\end{verbatim}

\begin{Shaded}
\begin{Highlighting}[]
\CommentTok{# #estratos}
\NormalTok{df}\OperatorTok{$}\NormalTok{distritoelec <-}\StringTok{ }\KeywordTok{paste}\NormalTok{(df}\OperatorTok{$}\NormalTok{ID_ESTADO, df}\OperatorTok{$}\NormalTok{ID_DISTRITO, df}\OperatorTok{$}\NormalTok{NOMBRE_DISTRITO, }\DataTypeTok{sep=}\StringTok{"-"}\NormalTok{)}
\CommentTok{# revisamos el tamaño deben ser  300 unicos}
\KeywordTok{length}\NormalTok{(}\KeywordTok{unique}\NormalTok{(df}\OperatorTok{$}\NormalTok{distritoelec))}
\end{Highlighting}
\end{Shaded}

\begin{verbatim}
## [1] 300
\end{verbatim}

\begin{Shaded}
\begin{Highlighting}[]
\NormalTok{midf2 <-}\StringTok{ }\NormalTok{df[, }\KeywordTok{c}\NormalTok{(}\StringTok{"distritoelec"}\NormalTok{, }\StringTok{"totalvotos"}\NormalTok{)]}
\NormalTok{midf2}\OperatorTok{$}\NormalTok{votos.favor <-}\StringTok{ }\NormalTok{midf}\OperatorTok{$}\NormalTok{votos.favor}

\CommentTok{# otorgamos id a cada distrito}
\NormalTok{iddistrito =}\StringTok{ }\DecValTok{1}\OperatorTok{:}\DecValTok{300}
\NormalTok{distritoelec =}\StringTok{ }\KeywordTok{unique}\NormalTok{(df}\OperatorTok{$}\NormalTok{distritoelec)}
\NormalTok{idsdist <-}\StringTok{ }\KeywordTok{data.frame}\NormalTok{(iddistrito, distritoelec)}
\NormalTok{midf2 <-}\StringTok{ }\KeywordTok{merge}\NormalTok{(midf2, idsdist)}

\CommentTok{# calculo de las probabilidades de inclusión para todos los elementos}
\NormalTok{prob.inc <-}\StringTok{ }\KeywordTok{data.frame}\NormalTok{(midf2 }\OperatorTok\StringTok{ }\KeywordTok{group_by}\NormalTok{(iddistrito) }\OperatorTok\StringTok{ }\KeywordTok{count}\NormalTok{())}
\NormalTok{prob.inc}\OperatorTok{$}\NormalTok{wk <-}\StringTok{ }\DecValTok{1}\OperatorTok{/}\NormalTok{(}\DecValTok{4}\OperatorTok{/}\NormalTok{prob.inc}\OperatorTok{$}\NormalTok{n)}

\NormalTok{midf2 <-}\StringTok{ }\KeywordTok{merge}\NormalTok{(midf2, prob.inc, }\DataTypeTok{by=}\StringTok{"iddistrito"}\NormalTok{)}

\CommentTok{# creamos un subid (id dentro del distrito electoral)}
\NormalTok{subid <-}\StringTok{ }\KeywordTok{c}\NormalTok{()}
\ControlFlowTok{for}\NormalTok{ (i }\ControlFlowTok{in} \DecValTok{1}\OperatorTok{:}\DecValTok{300}\NormalTok{)\{}
\NormalTok{  s0 =}\StringTok{ }\NormalTok{midf2}\OperatorTok{$}\NormalTok{distritoelec[midf2}\OperatorTok{$}\NormalTok{iddistrito }\OperatorTok{==}\StringTok{ }\NormalTok{i]}
\NormalTok{  subid =}\StringTok{ }\KeywordTok{c}\NormalTok{(subid, }\DecValTok{1}\OperatorTok{:}\KeywordTok{length}\NormalTok{(s0))}
\NormalTok{\}}
\NormalTok{midf2}\OperatorTok{$}\NormalTok{subid <-}\StringTok{ }\NormalTok{subid}

\CommentTok{# creamos un id único en toda la  ppoblacion}
\NormalTok{midf2}\OperatorTok{$}\NormalTok{idun <-}\StringTok{ }\KeywordTok{paste}\NormalTok{(midf2}\OperatorTok{$}\NormalTok{iddistrito, midf2}\OperatorTok{$}\NormalTok{subid, }\DataTypeTok{sep =} \StringTok{"-"}\NormalTok{)}


\CommentTok{# observemos que esta es la función que se repitara 1000 vveces para la simulación}
\CommentTok{# lo que reduce el tiempo de ejecución}
\NormalTok{disenost <-}\StringTok{ }\ControlFlowTok{function}\NormalTok{()\{}
  \CommentTok{#seleccionamos muestra con base en el iduniico}
\NormalTok{  idun <-}\StringTok{ }\KeywordTok{c}\NormalTok{()}
  \ControlFlowTok{for}\NormalTok{ (i }\ControlFlowTok{in} \DecValTok{1}\OperatorTok{:}\DecValTok{300}\NormalTok{) \{}
\NormalTok{    idun <-}\StringTok{ }\KeywordTok{c}\NormalTok{(idun, }\KeywordTok{paste}\NormalTok{(i, }\KeywordTok{sample}\NormalTok{(}\DecValTok{1}\OperatorTok{:}\NormalTok{prob.inc}\OperatorTok{$}\NormalTok{n[i], }\DataTypeTok{size =} \DecValTok{4}\NormalTok{), }\DataTypeTok{sep=}\StringTok{"-"}\NormalTok{))}
\NormalTok{  \}}
\NormalTok{  seleccion <-}\StringTok{ }\KeywordTok{data.frame}\NormalTok{(idun)}
  \CommentTok{#obtenemos los datos de la muestra}
\NormalTok{  muestra <-}\StringTok{ }\KeywordTok{left_join}\NormalTok{(seleccion, midf2, }\DataTypeTok{by =} \StringTok{"idun"}\NormalTok{)}
  
  \CommentTok{#esimaciones}
  \KeywordTok{return}\NormalTok{(}\KeywordTok{sum}\NormalTok{(muestra}\OperatorTok{$}\NormalTok{votos.favor}\OperatorTok{*}\NormalTok{muestra}\OperatorTok{$}\NormalTok{wk)}\OperatorTok{/}\KeywordTok{sum}\NormalTok{(muestra}\OperatorTok{$}\NormalTok{totalvotos}\OperatorTok{*}\NormalTok{muestra}\OperatorTok{$}\NormalTok{wk))}
\NormalTok{\}}
\NormalTok{simulacion.estr <-}\StringTok{ }\KeywordTok{replicate}\NormalTok{(}\DecValTok{1000}\NormalTok{, }\KeywordTok{disenost}\NormalTok{())}
\KeywordTok{mean}\NormalTok{(simulacion.estr)}\OperatorTok{*}\DecValTok{100}
\end{Highlighting}
\end{Shaded}

\begin{verbatim}
## [1] 44.27721997
\end{verbatim}

\begin{Shaded}
\begin{Highlighting}[]
\CommentTok{# trabjaremos con un dataframe con la siguiente}
\NormalTok{midf3 <-}\StringTok{ }\NormalTok{df[, }\KeywordTok{c}\NormalTok{(}\StringTok{"NOMBRE_ESTADO"}\NormalTok{, }\StringTok{"NOMBRE_DISTRITO"}\NormalTok{, }\StringTok{"ID_DISTRITO"}\NormalTok{, }\StringTok{"ID_ESTADO"}\NormalTok{,}
                \StringTok{"distritoelec"}\NormalTok{, }\StringTok{"totalvotos"}\NormalTok{)]}
\CommentTok{# recuperamos las estimaciones hechas anteriormente }
\NormalTok{midf3}\OperatorTok{$}\NormalTok{votos.favor <-}\StringTok{ }\NormalTok{midf}\OperatorTok{$}\NormalTok{votos.favor}

\CommentTok{# asignamos circunscripciones con respecto a los estados}
\NormalTok{circ <-}\StringTok{ }\KeywordTok{c}\NormalTok{(}\DecValTok{2}\NormalTok{, }\DecValTok{1}\NormalTok{, }\DecValTok{1}\NormalTok{, }\DecValTok{3}\NormalTok{, }\DecValTok{2}\NormalTok{, }\DecValTok{5}\NormalTok{, }\DecValTok{3}\NormalTok{, }\DecValTok{1}\NormalTok{, }\DecValTok{4}\NormalTok{, }\DecValTok{1}\NormalTok{, }\DecValTok{2}\NormalTok{, }\DecValTok{4}\NormalTok{, }\DecValTok{5}\NormalTok{, }\DecValTok{1}\NormalTok{, }\DecValTok{5}\NormalTok{, }\DecValTok{5}\NormalTok{, }\DecValTok{4}\NormalTok{, }\DecValTok{1}\NormalTok{, }\DecValTok{2}\NormalTok{, }\DecValTok{3}\NormalTok{, }\DecValTok{4}\NormalTok{, }
          \DecValTok{2}\NormalTok{, }\DecValTok{3}\NormalTok{, }\DecValTok{2}\NormalTok{, }\DecValTok{1}\NormalTok{, }\DecValTok{1}\NormalTok{, }\DecValTok{3}\NormalTok{, }\DecValTok{2}\NormalTok{, }\DecValTok{4}\NormalTok{, }\DecValTok{3}\NormalTok{, }\DecValTok{3}\NormalTok{, }\DecValTok{2}\NormalTok{)}
\CommentTok{# recuperamos los nombres de los estados y hacemos un dataframe con los estados}
\CommentTok{# y circunscripciones}
\NormalTok{NOMBRE_ESTADO <-}\StringTok{ }\KeywordTok{unique}\NormalTok{(df}\OperatorTok{$}\NormalTok{NOMBRE_ESTADO)}
\NormalTok{circdf <-}\StringTok{ }\KeywordTok{data.frame}\NormalTok{(circ, NOMBRE_ESTADO)}
\CommentTok{# unimos los dataframes y ahora tenemos la circunscripcion en el dataframe}
\NormalTok{midf3 <-}\StringTok{ }\KeywordTok{merge}\NormalTok{(circdf, midf3, }\DataTypeTok{by=}\StringTok{"NOMBRE_ESTADO"}\NormalTok{)}


\CommentTok{# otorgamos id unico a cada distrito}
\NormalTok{iddistrito =}\StringTok{ }\DecValTok{1}\OperatorTok{:}\DecValTok{300}
\NormalTok{distritoelec =}\StringTok{ }\KeywordTok{unique}\NormalTok{(df}\OperatorTok{$}\NormalTok{distritoelec)}
\NormalTok{idsdist <-}\StringTok{ }\KeywordTok{data.frame}\NormalTok{(iddistrito, distritoelec)}
\NormalTok{midf3 <-}\StringTok{ }\KeywordTok{merge}\NormalTok{(midf3, idsdist)}

\CommentTok{#agrupara cada estrato (hacemos un dataframe para cada circunscripción)}
\NormalTok{circ1 <-}\StringTok{ }\NormalTok{midf3 }\OperatorTok\StringTok{ }\KeywordTok{filter}\NormalTok{(circ }\OperatorTok{==}\StringTok{ }\DecValTok{1}\NormalTok{)}
\NormalTok{circ2 <-}\StringTok{ }\NormalTok{midf3 }\OperatorTok\StringTok{ }\KeywordTok{filter}\NormalTok{(circ }\OperatorTok{==}\StringTok{ }\DecValTok{2}\NormalTok{)}
\NormalTok{circ3 <-}\StringTok{ }\NormalTok{midf3 }\OperatorTok\StringTok{ }\KeywordTok{filter}\NormalTok{(circ }\OperatorTok{==}\StringTok{ }\DecValTok{3}\NormalTok{)}
\NormalTok{circ4 <-}\StringTok{ }\NormalTok{midf3 }\OperatorTok\StringTok{ }\KeywordTok{filter}\NormalTok{(circ }\OperatorTok{==}\StringTok{ }\DecValTok{4}\NormalTok{)}
\NormalTok{circ5 <-}\StringTok{ }\NormalTok{midf3 }\OperatorTok\StringTok{ }\KeywordTok{filter}\NormalTok{(circ }\OperatorTok{==}\StringTok{ }\DecValTok{5}\NormalTok{)}

\CommentTok{# ahora a cada circunscripción le calcularemos cuantas casillas casillas le }
\CommentTok{# pertencecen, tambien un sub-id (que sera un id dentro de cada distrito), }
\CommentTok{# vector que guardara los subids}
\NormalTok{subid <-}\StringTok{ }\KeywordTok{c}\NormalTok{()}
\CommentTok{# vector que guardara cuantas casillas le pertencen a cada distrito}
\NormalTok{npob <-}\StringTok{ }\KeywordTok{c}\NormalTok{()}
\ControlFlowTok{for}\NormalTok{ (i }\ControlFlowTok{in} \KeywordTok{unique}\NormalTok{(circ1}\OperatorTok{$}\NormalTok{iddistrito))\{}
  \CommentTok{# selecionamos distritos únicos por cirscunscripción }
\NormalTok{  s0 =}\StringTok{ }\NormalTok{circ1}\OperatorTok{$}\NormalTok{distritoelec[circ1}\OperatorTok{$}\NormalTok{iddistrito }\OperatorTok{==}\StringTok{ }\NormalTok{i]}
  \CommentTok{# agregamos el subid}
\NormalTok{  subid =}\StringTok{ }\KeywordTok{c}\NormalTok{(subid, }\DecValTok{1}\OperatorTok{:}\KeywordTok{length}\NormalTok{(s0))}
  \CommentTok{# total de casillas que le pertencen al distrito}
\NormalTok{  npob <-}\StringTok{ }\KeywordTok{c}\NormalTok{(npob, }\KeywordTok{rep}\NormalTok{(}\KeywordTok{length}\NormalTok{(s0), }\KeywordTok{length}\NormalTok{(s0)))}
\NormalTok{\}}
\CommentTok{# agregamos los resultados anteriores al dataframe}
\NormalTok{circ1}\OperatorTok{$}\NormalTok{subid <-}\StringTok{ }\NormalTok{subid}
\NormalTok{circ1}\OperatorTok{$}\NormalTok{npob <-}\StringTok{ }\NormalTok{npob}
\CommentTok{# calculamos los pesos para cada casilla}
\NormalTok{circ1}\OperatorTok{$}\NormalTok{wk <-}\StringTok{ }\DecValTok{1}\OperatorTok{/}\NormalTok{(}\DecValTok{8}\OperatorTok{/}\KeywordTok{length}\NormalTok{(}\KeywordTok{unique}\NormalTok{(circ1}\OperatorTok{$}\NormalTok{iddistrito))}\OperatorTok{*}\DecValTok{30}\OperatorTok{/}\NormalTok{circ1}\OperatorTok{$}\NormalTok{npob)}
\CommentTok{# creamos un idunico para cada casilla con base en iddistrito y subid}
\NormalTok{circ1}\OperatorTok{$}\NormalTok{idun <-}\StringTok{ }\KeywordTok{paste}\NormalTok{(circ1}\OperatorTok{$}\NormalTok{iddistrito, circ1}\OperatorTok{$}\NormalTok{subid, }\DataTypeTok{sep=}\StringTok{"-"}\NormalTok{)}
\CommentTok{# relaciona el numero de distrito con el numero de casillas}
\NormalTok{prob.c1 <-}\StringTok{ }\KeywordTok{data.frame}\NormalTok{(circ1 }\OperatorTok\StringTok{ }\KeywordTok{group_by}\NormalTok{(iddistrito) }\OperatorTok\StringTok{ }\KeywordTok{count}\NormalTok{())}
\CommentTok{#repetimos este proceso para cada circunscripción}

\CommentTok{#circunscripción 2}
\NormalTok{subid <-}\StringTok{ }\KeywordTok{c}\NormalTok{()}
\NormalTok{npob <-}\StringTok{ }\KeywordTok{c}\NormalTok{()}
\ControlFlowTok{for}\NormalTok{ (i }\ControlFlowTok{in} \KeywordTok{unique}\NormalTok{(circ2}\OperatorTok{$}\NormalTok{iddistrito))\{}
\NormalTok{  s0 =}\StringTok{ }\NormalTok{circ2}\OperatorTok{$}\NormalTok{distritoelec[circ2}\OperatorTok{$}\NormalTok{iddistrito }\OperatorTok{==}\StringTok{ }\NormalTok{i]}
\NormalTok{  subid =}\StringTok{ }\KeywordTok{c}\NormalTok{(subid, }\DecValTok{1}\OperatorTok{:}\KeywordTok{length}\NormalTok{(s0))}
\NormalTok{  npob <-}\StringTok{ }\KeywordTok{c}\NormalTok{(npob, }\KeywordTok{rep}\NormalTok{(}\KeywordTok{length}\NormalTok{(s0), }\KeywordTok{length}\NormalTok{(s0)))}
\NormalTok{\}}
\NormalTok{circ2}\OperatorTok{$}\NormalTok{subid <-}\StringTok{ }\NormalTok{subid}
\NormalTok{circ2}\OperatorTok{$}\NormalTok{npob <-}\StringTok{ }\NormalTok{npob}
\NormalTok{circ2}\OperatorTok{$}\NormalTok{wk <-}\StringTok{ }\DecValTok{1}\OperatorTok{/}\NormalTok{(}\DecValTok{8}\OperatorTok{/}\KeywordTok{length}\NormalTok{(}\KeywordTok{unique}\NormalTok{(circ2}\OperatorTok{$}\NormalTok{iddistrito))}\OperatorTok{*}\DecValTok{30}\OperatorTok{/}\NormalTok{circ2}\OperatorTok{$}\NormalTok{npob)}
\NormalTok{prob.c2 <-}\StringTok{ }\KeywordTok{data.frame}\NormalTok{(circ2 }\OperatorTok\StringTok{ }\KeywordTok{group_by}\NormalTok{(iddistrito) }\OperatorTok\StringTok{ }\KeywordTok{count}\NormalTok{())}
\NormalTok{circ2}\OperatorTok{$}\NormalTok{idun <-}\StringTok{ }\KeywordTok{paste}\NormalTok{(circ2}\OperatorTok{$}\NormalTok{iddistrito, circ2}\OperatorTok{$}\NormalTok{subid, }\DataTypeTok{sep=}\StringTok{"-"}\NormalTok{)}

\CommentTok{# circunscripción 3}
\NormalTok{subid <-}\StringTok{ }\KeywordTok{c}\NormalTok{()}
\NormalTok{npob <-}\StringTok{ }\KeywordTok{c}\NormalTok{()}
\ControlFlowTok{for}\NormalTok{ (i }\ControlFlowTok{in} \KeywordTok{unique}\NormalTok{(circ3}\OperatorTok{$}\NormalTok{iddistrito))\{}
\NormalTok{  s0 =}\StringTok{ }\NormalTok{circ3}\OperatorTok{$}\NormalTok{distritoelec[circ3}\OperatorTok{$}\NormalTok{iddistrito }\OperatorTok{==}\StringTok{ }\NormalTok{i]}
\NormalTok{  subid =}\StringTok{ }\KeywordTok{c}\NormalTok{(subid, }\DecValTok{1}\OperatorTok{:}\KeywordTok{length}\NormalTok{(s0))}
\NormalTok{  npob <-}\StringTok{ }\KeywordTok{c}\NormalTok{(npob, }\KeywordTok{rep}\NormalTok{(}\KeywordTok{length}\NormalTok{(s0), }\KeywordTok{length}\NormalTok{(s0)))}
\NormalTok{\}}
\NormalTok{circ3}\OperatorTok{$}\NormalTok{subid <-}\StringTok{ }\NormalTok{subid}
\NormalTok{circ3}\OperatorTok{$}\NormalTok{npob <-}\StringTok{ }\NormalTok{npob}
\NormalTok{circ3}\OperatorTok{$}\NormalTok{wk <-}\StringTok{ }\DecValTok{1}\OperatorTok{/}\NormalTok{(}\DecValTok{8}\OperatorTok{/}\KeywordTok{length}\NormalTok{(}\KeywordTok{unique}\NormalTok{(circ3}\OperatorTok{$}\NormalTok{iddistrito))}\OperatorTok{*}\DecValTok{30}\OperatorTok{/}\NormalTok{circ3}\OperatorTok{$}\NormalTok{npob)}
\NormalTok{prob.c3 <-}\StringTok{ }\KeywordTok{data.frame}\NormalTok{(circ3 }\OperatorTok\StringTok{ }\KeywordTok{group_by}\NormalTok{(iddistrito) }\OperatorTok\StringTok{ }\KeywordTok{count}\NormalTok{())}
\NormalTok{circ3}\OperatorTok{$}\NormalTok{idun <-}\StringTok{ }\KeywordTok{paste}\NormalTok{(circ3}\OperatorTok{$}\NormalTok{iddistrito, circ3}\OperatorTok{$}\NormalTok{subid, }\DataTypeTok{sep=}\StringTok{"-"}\NormalTok{)}

\CommentTok{# circunscripcion 4}
\NormalTok{subid <-}\StringTok{ }\KeywordTok{c}\NormalTok{()}
\NormalTok{npob <-}\StringTok{ }\KeywordTok{c}\NormalTok{()}
\ControlFlowTok{for}\NormalTok{ (i }\ControlFlowTok{in} \KeywordTok{unique}\NormalTok{(circ4}\OperatorTok{$}\NormalTok{iddistrito))\{}
\NormalTok{  s0 =}\StringTok{ }\NormalTok{circ4}\OperatorTok{$}\NormalTok{distritoelec[circ4}\OperatorTok{$}\NormalTok{iddistrito }\OperatorTok{==}\StringTok{ }\NormalTok{i]}
\NormalTok{  subid =}\StringTok{ }\KeywordTok{c}\NormalTok{(subid, }\DecValTok{1}\OperatorTok{:}\KeywordTok{length}\NormalTok{(s0))}
\NormalTok{  npob <-}\StringTok{ }\KeywordTok{c}\NormalTok{(npob, }\KeywordTok{rep}\NormalTok{(}\KeywordTok{length}\NormalTok{(s0), }\KeywordTok{length}\NormalTok{(s0)))}
\NormalTok{\}}
\NormalTok{circ4}\OperatorTok{$}\NormalTok{subid <-}\StringTok{ }\NormalTok{subid}
\NormalTok{circ4}\OperatorTok{$}\NormalTok{npob <-}\StringTok{ }\NormalTok{npob}
\NormalTok{circ4}\OperatorTok{$}\NormalTok{wk <-}\StringTok{ }\DecValTok{1}\OperatorTok{/}\NormalTok{(}\DecValTok{8}\OperatorTok{/}\KeywordTok{length}\NormalTok{(}\KeywordTok{unique}\NormalTok{(circ4}\OperatorTok{$}\NormalTok{iddistrito))}\OperatorTok{*}\DecValTok{30}\OperatorTok{/}\NormalTok{circ4}\OperatorTok{$}\NormalTok{npob)}
\NormalTok{prob.c4 <-}\StringTok{ }\KeywordTok{data.frame}\NormalTok{(circ4 }\OperatorTok\StringTok{ }\KeywordTok{group_by}\NormalTok{(iddistrito) }\OperatorTok\StringTok{ }\KeywordTok{count}\NormalTok{())}
\NormalTok{circ4}\OperatorTok{$}\NormalTok{idun <-}\StringTok{ }\KeywordTok{paste}\NormalTok{(circ4}\OperatorTok{$}\NormalTok{iddistrito, circ4}\OperatorTok{$}\NormalTok{subid, }\DataTypeTok{sep=}\StringTok{"-"}\NormalTok{)}

\CommentTok{# circunscripcion 5}
\NormalTok{subid <-}\StringTok{ }\KeywordTok{c}\NormalTok{()}
\NormalTok{npob <-}\StringTok{ }\KeywordTok{c}\NormalTok{()}
\ControlFlowTok{for}\NormalTok{ (i }\ControlFlowTok{in} \KeywordTok{unique}\NormalTok{(circ5}\OperatorTok{$}\NormalTok{iddistrito))\{}
\NormalTok{  s0 =}\StringTok{ }\NormalTok{circ5}\OperatorTok{$}\NormalTok{distritoelec[circ5}\OperatorTok{$}\NormalTok{iddistrito }\OperatorTok{==}\StringTok{ }\NormalTok{i]}
\NormalTok{  subid =}\StringTok{ }\KeywordTok{c}\NormalTok{(subid, }\DecValTok{1}\OperatorTok{:}\KeywordTok{length}\NormalTok{(s0))}
\NormalTok{  npob <-}\StringTok{ }\KeywordTok{c}\NormalTok{(npob, }\KeywordTok{rep}\NormalTok{(}\KeywordTok{length}\NormalTok{(s0), }\KeywordTok{length}\NormalTok{(s0)))}
\NormalTok{\}}
\NormalTok{circ5}\OperatorTok{$}\NormalTok{subid <-}\StringTok{ }\NormalTok{subid}
\NormalTok{circ5}\OperatorTok{$}\NormalTok{npob <-}\StringTok{ }\NormalTok{npob}
\NormalTok{circ5}\OperatorTok{$}\NormalTok{wk <-}\StringTok{ }\DecValTok{1}\OperatorTok{/}\NormalTok{(}\DecValTok{8}\OperatorTok{/}\KeywordTok{length}\NormalTok{(}\KeywordTok{unique}\NormalTok{(circ5}\OperatorTok{$}\NormalTok{iddistrito))}\OperatorTok{*}\DecValTok{30}\OperatorTok{/}\NormalTok{circ5}\OperatorTok{$}\NormalTok{npob)}
\NormalTok{prob.c5 <-}\StringTok{ }\KeywordTok{data.frame}\NormalTok{(circ5 }\OperatorTok\StringTok{ }\KeywordTok{group_by}\NormalTok{(iddistrito) }\OperatorTok\StringTok{ }\KeywordTok{count}\NormalTok{())}
\NormalTok{circ5}\OperatorTok{$}\NormalTok{idun <-}\StringTok{ }\KeywordTok{paste}\NormalTok{(circ5}\OperatorTok{$}\NormalTok{iddistrito, circ5}\OperatorTok{$}\NormalTok{subid, }\DataTypeTok{sep=}\StringTok{"-"}\NormalTok{)}



\NormalTok{diseno.est.c1 <-}\StringTok{ }\ControlFlowTok{function}\NormalTok{()\{}
  \CommentTok{# seleccionamos 8 distritos de la circunscripción}
\NormalTok{  seleccion <-}\StringTok{  }\KeywordTok{sample}\NormalTok{(}\KeywordTok{unique}\NormalTok{(circ1}\OperatorTok{$}\NormalTok{iddistrito), }\DataTypeTok{size=}\DecValTok{8}\NormalTok{)}
  \CommentTok{# aquí guardaremos el idunico de las casillas seleccions en la muestra}
\NormalTok{  idun <-}\StringTok{ }\KeywordTok{c}\NormalTok{()}
  \CommentTok{# para cada distrito de los 8 seleccionados}
  \ControlFlowTok{for}\NormalTok{ (i }\ControlFlowTok{in}\NormalTok{ seleccion) \{}
    \CommentTok{# vemos cuantas casillas le pertenecen}
\NormalTok{    iddist <-}\StringTok{ }\NormalTok{prob.c1 }\OperatorTok\StringTok{ }\KeywordTok{filter}\NormalTok{(iddistrito }\OperatorTok{==}\StringTok{ }\NormalTok{i)}
    \CommentTok{# y seleccionamos 30 elemento con un m.a.s.}
\NormalTok{    idun <-}\StringTok{ }\KeywordTok{c}\NormalTok{(idun, }\KeywordTok{paste}\NormalTok{(i, }\KeywordTok{sample}\NormalTok{(iddist}\OperatorTok{$}\NormalTok{n, }\DataTypeTok{size =} \DecValTok{30}\NormalTok{), }\DataTypeTok{sep=}\StringTok{"-"}\NormalTok{))}
\NormalTok{  \}}
  \CommentTok{#obtenemos los datos de la muestra}
\NormalTok{  muestra.c1 <-}\StringTok{ }\KeywordTok{left_join}\NormalTok{(}\KeywordTok{data.frame}\NormalTok{(idun), circ1, }\DataTypeTok{by =} \StringTok{"idun"}\NormalTok{)}
  \CommentTok{# recuperamos la muestra de esta circunscripción}
  \KeywordTok{return}\NormalTok{(muestra.c1)}
\NormalTok{\}}
\CommentTok{# de manera analoga para el resto de circunscripciones}

\CommentTok{#circunscripcion 2}
\NormalTok{diseno.est.c2 <-}\StringTok{ }\ControlFlowTok{function}\NormalTok{()\{}
\NormalTok{  seleccion <-}\StringTok{  }\KeywordTok{sample}\NormalTok{(}\KeywordTok{unique}\NormalTok{(circ2}\OperatorTok{$}\NormalTok{iddistrito), }\DataTypeTok{size=}\DecValTok{8}\NormalTok{)}
\NormalTok{  idun <-}\StringTok{ }\KeywordTok{c}\NormalTok{()}
  \ControlFlowTok{for}\NormalTok{ (i }\ControlFlowTok{in}\NormalTok{ seleccion) \{}
\NormalTok{    iddist <-}\StringTok{ }\NormalTok{prob.c2 }\OperatorTok\StringTok{ }\KeywordTok{filter}\NormalTok{(iddistrito }\OperatorTok{==}\StringTok{ }\NormalTok{i)}
\NormalTok{    idun <-}\StringTok{ }\KeywordTok{c}\NormalTok{(idun, }\KeywordTok{paste}\NormalTok{(i, }\KeywordTok{sample}\NormalTok{(iddist}\OperatorTok{$}\NormalTok{n, }\DataTypeTok{size =} \DecValTok{30}\NormalTok{), }\DataTypeTok{sep=}\StringTok{"-"}\NormalTok{))}
\NormalTok{  \}}
\NormalTok{  muestra.c2 <-}\StringTok{ }\KeywordTok{left_join}\NormalTok{(}\KeywordTok{data.frame}\NormalTok{(idun), circ2, }\DataTypeTok{by =} \StringTok{"idun"}\NormalTok{)}
  
  \KeywordTok{return}\NormalTok{(muestra.c2)}
\NormalTok{\}}

\CommentTok{#circunscrición 3}
\NormalTok{diseno.est.c3 <-}\StringTok{ }\ControlFlowTok{function}\NormalTok{()\{}
\NormalTok{  seleccion <-}\StringTok{  }\KeywordTok{sample}\NormalTok{(}\KeywordTok{unique}\NormalTok{(circ3}\OperatorTok{$}\NormalTok{iddistrito), }\DataTypeTok{size=}\DecValTok{8}\NormalTok{)}
\NormalTok{  idun <-}\StringTok{ }\KeywordTok{c}\NormalTok{()}
  \ControlFlowTok{for}\NormalTok{ (i }\ControlFlowTok{in}\NormalTok{ seleccion) \{}
\NormalTok{    iddist <-}\StringTok{ }\NormalTok{prob.c3 }\OperatorTok\StringTok{ }\KeywordTok{filter}\NormalTok{(iddistrito }\OperatorTok{==}\StringTok{ }\NormalTok{i)}
\NormalTok{    idun <-}\StringTok{ }\KeywordTok{c}\NormalTok{(idun, }\KeywordTok{paste}\NormalTok{(i, }\KeywordTok{sample}\NormalTok{(iddist}\OperatorTok{$}\NormalTok{n, }\DataTypeTok{size =} \DecValTok{30}\NormalTok{), }\DataTypeTok{sep=}\StringTok{"-"}\NormalTok{))}
\NormalTok{  \}}
\NormalTok{  muestra.c3 <-}\StringTok{ }\KeywordTok{left_join}\NormalTok{(}\KeywordTok{data.frame}\NormalTok{(idun), circ3, }\DataTypeTok{by =} \StringTok{"idun"}\NormalTok{)}
  
  \KeywordTok{return}\NormalTok{(muestra.c3)}
\NormalTok{\}}

\CommentTok{#circunscripción 4}
\NormalTok{diseno.est.c4 <-}\StringTok{ }\ControlFlowTok{function}\NormalTok{()\{}
\NormalTok{  seleccion <-}\StringTok{  }\KeywordTok{sample}\NormalTok{(}\KeywordTok{unique}\NormalTok{(circ4}\OperatorTok{$}\NormalTok{iddistrito), }\DataTypeTok{size=}\DecValTok{8}\NormalTok{)}
\NormalTok{  idun <-}\StringTok{ }\KeywordTok{c}\NormalTok{()}
  \ControlFlowTok{for}\NormalTok{ (i }\ControlFlowTok{in}\NormalTok{ seleccion) \{}
\NormalTok{    iddist <-}\StringTok{ }\NormalTok{prob.c4 }\OperatorTok\StringTok{ }\KeywordTok{filter}\NormalTok{(iddistrito }\OperatorTok{==}\StringTok{ }\NormalTok{i)}
\NormalTok{    idun <-}\StringTok{ }\KeywordTok{c}\NormalTok{(idun, }\KeywordTok{paste}\NormalTok{(i, }\KeywordTok{sample}\NormalTok{(iddist}\OperatorTok{$}\NormalTok{n, }\DataTypeTok{size =} \DecValTok{30}\NormalTok{), }\DataTypeTok{sep=}\StringTok{"-"}\NormalTok{))}
\NormalTok{  \}}
\NormalTok{  muestra.c4 <-}\StringTok{ }\KeywordTok{left_join}\NormalTok{(}\KeywordTok{data.frame}\NormalTok{(idun), circ4, }\DataTypeTok{by =} \StringTok{"idun"}\NormalTok{)}
  
  \KeywordTok{return}\NormalTok{(muestra.c4)}
\NormalTok{\}}


\CommentTok{#circunscripción 5}
\NormalTok{diseno.est.c5 <-}\StringTok{ }\ControlFlowTok{function}\NormalTok{()\{}
\NormalTok{  seleccion <-}\StringTok{  }\KeywordTok{sample}\NormalTok{(}\KeywordTok{unique}\NormalTok{(circ5}\OperatorTok{$}\NormalTok{iddistrito), }\DataTypeTok{size=}\DecValTok{8}\NormalTok{)}
\NormalTok{  idun <-}\StringTok{ }\KeywordTok{c}\NormalTok{()}
  \ControlFlowTok{for}\NormalTok{ (i }\ControlFlowTok{in}\NormalTok{ seleccion) \{}
\NormalTok{    iddist <-}\StringTok{ }\NormalTok{prob.c5 }\OperatorTok\StringTok{ }\KeywordTok{filter}\NormalTok{(iddistrito }\OperatorTok{==}\StringTok{ }\NormalTok{i)}
\NormalTok{    idun <-}\StringTok{ }\KeywordTok{c}\NormalTok{(idun, }\KeywordTok{paste}\NormalTok{(i, }\KeywordTok{sample}\NormalTok{(iddist}\OperatorTok{$}\NormalTok{n, }\DataTypeTok{size =} \DecValTok{30}\NormalTok{), }\DataTypeTok{sep=}\StringTok{"-"}\NormalTok{))}
\NormalTok{  \}}
  \CommentTok{#obtenemos los datos de la muestra}
\NormalTok{  muestra.c5 <-}\StringTok{ }\KeywordTok{left_join}\NormalTok{(}\KeywordTok{data.frame}\NormalTok{(idun), circ5, }\DataTypeTok{by =} \StringTok{"idun"}\NormalTok{)}
  
  \KeywordTok{return}\NormalTok{(muestra.c5)}
\NormalTok{\}}

\NormalTok{diseno.circs <-}\StringTok{ }\ControlFlowTok{function}\NormalTok{()\{}
\NormalTok{  c1 <-}\StringTok{ }\KeywordTok{diseno.est.c1}\NormalTok{()}
\NormalTok{  c2 <-}\StringTok{ }\KeywordTok{diseno.est.c2}\NormalTok{()}
\NormalTok{  c3 <-}\StringTok{ }\KeywordTok{diseno.est.c3}\NormalTok{()}
\NormalTok{  c4 <-}\StringTok{ }\KeywordTok{diseno.est.c4}\NormalTok{()}
\NormalTok{  c5 <-}\StringTok{ }\KeywordTok{diseno.est.c5}\NormalTok{()}
  
\NormalTok{  muestra.circs <-}\StringTok{ }\KeywordTok{union}\NormalTok{(c1,c2)}
\NormalTok{  muestra.circs <-}\StringTok{ }\KeywordTok{union}\NormalTok{(muestra.circs,c3)}
\NormalTok{  muestra.circs <-}\StringTok{ }\KeywordTok{union}\NormalTok{(muestra.circs,c4)}
\NormalTok{  muestra.circs <-}\StringTok{ }\KeywordTok{union}\NormalTok{(muestra.circs,c5)}
  
  \KeywordTok{return}\NormalTok{(}\KeywordTok{sum}\NormalTok{(muestra.circs}\OperatorTok{$}\NormalTok{votos.favor}\OperatorTok{*}\NormalTok{muestra.circs}\OperatorTok{$}\NormalTok{wk)}\OperatorTok{/}\KeywordTok{sum}\NormalTok{(muestra.circs}\OperatorTok{$}\NormalTok{totalvotos}\OperatorTok{*}\NormalTok{muestra.circs}\OperatorTok{$}\NormalTok{wk))}
\NormalTok{\}}

\NormalTok{simulacion.circs <-}\StringTok{ }\KeywordTok{replicate}\NormalTok{(}\DecValTok{1000}\NormalTok{, }\KeywordTok{diseno.circs}\NormalTok{())}
\KeywordTok{mean}\NormalTok{(simulacion.circs)}\OperatorTok{*}\DecValTok{100}
\end{Highlighting}
\end{Shaded}

\begin{verbatim}
## [1] 44.26977854
\end{verbatim}

\begin{Shaded}
\begin{Highlighting}[]
\CommentTok{# ahora calculamos los errores cuadráticos medios de cada diseño}

\CommentTok{# del diseño 1}
\KeywordTok{mean}\NormalTok{((}\DecValTok{100}\OperatorTok{*}\NormalTok{simulacion.mas }\OperatorTok{-}\StringTok{ }\NormalTok{prct.votos)}\OperatorTok{^}\DecValTok{2}\NormalTok{)}
\end{Highlighting}
\end{Shaded}

\begin{verbatim}
## [1] 0.2990906677
\end{verbatim}

\begin{Shaded}
\begin{Highlighting}[]
\CommentTok{# del diseño 2}
\KeywordTok{mean}\NormalTok{((}\DecValTok{100}\OperatorTok{*}\NormalTok{simulacion.estr }\OperatorTok{-}\StringTok{ }\NormalTok{prct.votos)}\OperatorTok{^}\DecValTok{2}\NormalTok{)}
\end{Highlighting}
\end{Shaded}

\begin{verbatim}
## [1] 0.1342909203
\end{verbatim}

\begin{Shaded}
\begin{Highlighting}[]
\CommentTok{# del diseño 3}
\KeywordTok{mean}\NormalTok{((}\DecValTok{100}\OperatorTok{*}\NormalTok{simulacion.circs }\OperatorTok{-}\StringTok{ }\NormalTok{prct.votos)}\OperatorTok{^}\DecValTok{2}\NormalTok{)}
\end{Highlighting}
\end{Shaded}

\begin{verbatim}
## [1] 2.842545481
\end{verbatim}

\begin{Shaded}
\begin{Highlighting}[]
\KeywordTok{hist}\NormalTok{(}\DecValTok{100}\OperatorTok{*}\NormalTok{simulacion.mas, }\DataTypeTok{freq =}\NormalTok{ F, }\DataTypeTok{breaks =} \DecValTok{50}\NormalTok{, }
     \DataTypeTok{main =} \StringTok{"Simulación muestro aleatorio simple"}\NormalTok{,}
     \DataTypeTok{xlab =} \StringTok{"Porcentaje de votos a favor"}\NormalTok{, }\DataTypeTok{ylab =} \StringTok{"Frecuencia relativa"}\NormalTok{)}
\KeywordTok{abline}\NormalTok{(}\DataTypeTok{v =}\NormalTok{ prct.votos, }\DataTypeTok{add =}\NormalTok{ T, }\DataTypeTok{col=}\DecValTok{2}\NormalTok{)}
\end{Highlighting}
\end{Shaded}

\includegraphics{ex-03_files/figure-latex/unnamed-chunk-10-1.pdf}

\begin{Shaded}
\begin{Highlighting}[]
\KeywordTok{hist}\NormalTok{(}\DecValTok{100}\OperatorTok{*}\NormalTok{simulacion.estr, }\DataTypeTok{freq =}\NormalTok{ F, }\DataTypeTok{breaks =} \DecValTok{50}\NormalTok{, }
     \DataTypeTok{main =} \StringTok{"Simulación muestro estratificado"}\NormalTok{,}
     \DataTypeTok{xlab =} \StringTok{"Porcentaje votos a favor"}\NormalTok{, }\DataTypeTok{ylab=}\StringTok{"Frecuencia relativa"}\NormalTok{)}
\KeywordTok{abline}\NormalTok{(}\DataTypeTok{v =}\NormalTok{ prct.votos, }\DataTypeTok{add =}\NormalTok{ T, }\DataTypeTok{col=}\DecValTok{2}\NormalTok{)}
\end{Highlighting}
\end{Shaded}

\includegraphics{ex-03_files/figure-latex/unnamed-chunk-10-2.pdf}

\begin{Shaded}
\begin{Highlighting}[]
\KeywordTok{hist}\NormalTok{(}\DecValTok{100}\OperatorTok{*}\NormalTok{simulacion.circs, }\DataTypeTok{freq =}\NormalTok{ F, }\DataTypeTok{breaks =} \DecValTok{50}\NormalTok{,}
     \DataTypeTok{main =} \StringTok{"Simulacion muestreo estratificado y bietápico"}\NormalTok{,}
     \DataTypeTok{xlab =} \StringTok{"Porcentaje votos a favor"}\NormalTok{, }\DataTypeTok{ylab =} \StringTok{"Frecuencia relativa"}\NormalTok{)}
\KeywordTok{abline}\NormalTok{(}\DataTypeTok{v =}\NormalTok{ prct.votos, }\DataTypeTok{add =}\NormalTok{ T, }\DataTypeTok{col=}\DecValTok{2}\NormalTok{)}
\end{Highlighting}
\end{Shaded}

\includegraphics{ex-03_files/figure-latex/unnamed-chunk-10-3.pdf} De lo
anterior vemos que el diseño que mejor funciona es el diseño dos, esto
es porque a cada distrito le estamos tomando muestra, a diferencia del
diseño tres donde solo le tomamos muestra a 40 de 300 distritos, esto
hace que sea el peor, sin embargo en cuestiones de costo, puede llegar a
ser el más económico, finalmente el diseño que esta en medio es el
aleatorio simple.

\hypertarget{estimaciuxf3n-buxe1sica-de-una-encuesta-con-diseuxf1o-multietuxe1pico}{%
\subsection{3 Estimación básica de una encuesta con diseño
multietápico}\label{estimaciuxf3n-buxe1sica-de-una-encuesta-con-diseuxf1o-multietuxe1pico}}

Considere la Encuesta Nacional de Vivienda (ENVI) 2020

\url{https://www.inegi.org.mx/programas/envi/2020/}

Suponga que será el encargado de generar los resultados básicos
presentados en el tabulado llamado Cuadro 5.1, ver Figura \ref{Envi2020}
(\url{https://www.inegi.org.mx/contenidos/programas/envi/2020/tabulados/envi_2020_tema_05_xlsx.zip})

En particular realice lo siguiente

\begin{enumerate}
\def\labelenumi{\roman{enumi}.}
\tightlist
\item
  Describa brevemente el diseño de muestreo usado en la encuesta. Es
  decir, si es muestreo aleatorio simple, tiene estratificación, es por
  conglomerados, etc.
\end{enumerate}

Se usó un diseño probabilístico, cuyo diseño muestral fue bietápico,
donde en la primera etapa se usó un diseño de muestreo estratificado y
en la segunda etapa un diseño de muestreo por conglomerados.

\begin{enumerate}
\def\labelenumi{\roman{enumi}.}
\tightlist
\item
  Identifique las variables asociadas al diseño de muestreo que están
  presentes en la base de datos a usar (THOGAR en
  \url{https://www.inegi.org.mx/programas/envi/2020/\#Microdatos})
\item
  Identifique la pregunta y variable asociada a la identificación de los
  Hogares con necesidad de rentar, comprar o construir una vivienda
  independiente de la que habitan.
\item
  Con esta información, estime el número total de hogares y el
  porcentaje de hogares que tienen una necesidad de vivienda a nivel
  nacional y por entidad federativa.
\item
  Calcule intervalos de confianza para los parámetros estimados en el
  inciso anterior. Comente sobre los resultados obtenidos.
\item
  \textbf{Punto extra opcional}. Considerando el porcentaje de hogares
  que tienen una necesidad de vivienda por entidad federativa, realice
  un mapa de calor (Geographic Heat Map) y comente los resultados.
\end{enumerate}

\begin{figure}[H]
\centering
    \includegraphics[width=160mm]{images/ENVI2020.png}
    \caption{Parte del cuadro 5.1 de los resultados de la ENVI 2020}
    \label{Envi2020}
\end{figure}

\begin{figure}[H]
\centering
    \includegraphics[width=160mm]{images/enc.png}
    \caption{Pregunta de interés, demandas y necesidades de vivienda}
    \label{Pregunta de interés}
\end{figure}

\begin{Shaded}
\begin{Highlighting}[]
\CommentTok{# leemos la base de datos}
\KeywordTok{setwd}\NormalTok{(}\StringTok{"D:/Notas/Muestreo/Exámen/examen02/Bases de datos"}\NormalTok{)}

\NormalTok{thogar <-}\StringTok{ }\NormalTok{data.table}\OperatorTok{::}\KeywordTok{fread}\NormalTok{(}\StringTok{"THOGAR.csv"}\NormalTok{)}

\CommentTok{# P3A1_1 es la variable a la que nos interesa replicar la estimación}
\CommentTok{# primero observamos que no hay N.A. por lo cual no es necesario hacer correciones}
\NormalTok{thogar2 <-}\StringTok{ }\NormalTok{thogar }\OperatorTok\StringTok{ }\KeywordTok{drop_na}\NormalTok{(P3A1_}\DecValTok{1}\NormalTok{)}
\KeywordTok{length}\NormalTok{(thogar2}\OperatorTok{$}\NormalTok{P3A1_}\DecValTok{1}\NormalTok{) }\OperatorTok{==}\StringTok{ }\KeywordTok{length}\NormalTok{(thogar}\OperatorTok{$}\NormalTok{P3A1_}\DecValTok{1}\NormalTok{)}
\end{Highlighting}
\end{Shaded}

\begin{verbatim}
## [1] TRUE
\end{verbatim}

\begin{Shaded}
\begin{Highlighting}[]
\CommentTok{# vemos las variables que usaremos en el diseño muestral}
\KeywordTok{summary}\NormalTok{(thogar[, }\KeywordTok{c}\NormalTok{(}\StringTok{"UPM_DIS"}\NormalTok{, }\StringTok{"EST_DIS"}\NormalTok{, }\StringTok{"FACTOR"}\NormalTok{)])}
\end{Highlighting}
\end{Shaded}

\begin{verbatim}
##     UPM_DIS            EST_DIS             FACTOR         
##  Min.   :   1.000   Min.   :  1.0000   Min.   :   4.0000  
##  1st Qu.:2254.000   1st Qu.:135.0000   1st Qu.: 254.0000  
##  Median :4291.500   Median :268.0000   Median : 473.0000  
##  Mean   :4306.808   Mean   :267.6786   Mean   : 640.0097  
##  3rd Qu.:6329.000   3rd Qu.:397.0000   3rd Qu.: 788.0000  
##  Max.   :8301.000   Max.   :552.0000   Max.   :8610.0000
\end{verbatim}

\begin{Shaded}
\begin{Highlighting}[]
\CommentTok{#los que si tienen necesidad}
\CommentTok{# segun la estructura de archivo}
\CommentTok{# el número 1 corresponde a si tienen necesidad}
\NormalTok{a <-}\StringTok{ }\KeywordTok{sum}\NormalTok{(thogar}\OperatorTok{$}\NormalTok{P3A1_}\DecValTok{1} \OperatorTok{==}\StringTok{ }\DecValTok{1}\NormalTok{)}
\CommentTok{# el número 2 a los que no}
\NormalTok{b <-}\StringTok{ }\KeywordTok{sum}\NormalTok{(thogar}\OperatorTok{$}\NormalTok{P3A1_}\DecValTok{1} \OperatorTok{==}\StringTok{ }\DecValTok{2}\NormalTok{)}
\CommentTok{# el número 3 a los que no especificaron}
\NormalTok{c <-}\StringTok{ }\KeywordTok{sum}\NormalTok{(thogar}\OperatorTok{$}\NormalTok{P3A1_}\DecValTok{1} \OperatorTok{==}\StringTok{ }\DecValTok{9}\NormalTok{)}
\CommentTok{# efectivamente estos son los únicos resultados}
\KeywordTok{sum}\NormalTok{(a }\OperatorTok{+}\StringTok{ }\NormalTok{b }\OperatorTok{+}\StringTok{ }\NormalTok{c) }\OperatorTok{==}\StringTok{ }\KeywordTok{length}\NormalTok{(thogar}\OperatorTok{$}\NormalTok{P3A1_}\DecValTok{1}\NormalTok{)}
\end{Highlighting}
\end{Shaded}

\begin{verbatim}
## [1] TRUE
\end{verbatim}

\begin{Shaded}
\begin{Highlighting}[]
\CommentTok{# añadimos estos valores al dataframe}
\CommentTok{# los que tienen necesida}
\NormalTok{thogar}\OperatorTok{$}\NormalTok{si <-}\StringTok{ }\KeywordTok{as.numeric}\NormalTok{(thogar}\OperatorTok{$}\NormalTok{P3A1_}\DecValTok{1} \OperatorTok{==}\StringTok{ }\DecValTok{1}\NormalTok{)}
\CommentTok{# los que no tienen necesidad}
\NormalTok{thogar}\OperatorTok{$}\NormalTok{no <-}\StringTok{ }\KeywordTok{as.numeric}\NormalTok{(thogar}\OperatorTok{$}\NormalTok{P3A1_}\DecValTok{1} \OperatorTok{==}\StringTok{ }\DecValTok{2}\NormalTok{)}
\CommentTok{# los que no especificaran}
\NormalTok{thogar}\OperatorTok{$}\NormalTok{ne <-}\StringTok{ }\KeywordTok{as.numeric}\NormalTok{(thogar}\OperatorTok{$}\NormalTok{P3A1_}\DecValTok{1} \OperatorTok{==}\StringTok{ }\DecValTok{9}\NormalTok{)}
\CommentTok{# ocuparemos un vector de unos para calcular el total}
\NormalTok{thogar}\OperatorTok{$}\NormalTok{total <-}\StringTok{ }\DecValTok{1}

\KeywordTok{library}\NormalTok{(survey)}
\CommentTok{# en muchos casos sólo hay una upm en cada estrato, lo que }
\CommentTok{# dificulta la estimación de la varianza por esto ocupamos esta opción}
\KeywordTok{options}\NormalTok{(}\DataTypeTok{survey.lonely.psu=}\StringTok{"adjust"}\NormalTok{)}
\CommentTok{# usamos nest=TRUE ya que no hay seguridad de que las claves de las UPM son únicas}

\CommentTok{# definimos el diseño}
\NormalTok{dsg.envi <-}\StringTok{ }\KeywordTok{svydesign}\NormalTok{(}\DataTypeTok{id=}\OperatorTok{~}\NormalTok{UPM_DIS, }\DataTypeTok{strat=}\OperatorTok{~}\NormalTok{EST_DIS, }\DataTypeTok{weight =}\OperatorTok{~}\NormalTok{FACTOR,}
                      \DataTypeTok{data =}\NormalTok{ thogar, }\DataTypeTok{nest=}\NormalTok{T)}
\CommentTok{# summary(dsg.envi)}

\CommentTok{# guardamos los resultados}
\CommentTok{# por nivel nacional}
\NormalTok{rel.nac <-}\StringTok{ }\KeywordTok{svymean}\NormalTok{(}\OperatorTok{~}\NormalTok{si }\OperatorTok{+}\StringTok{ }\NormalTok{no }\OperatorTok{+}\StringTok{ }\NormalTok{ne, dsg.envi)}\OperatorTok{*}\DecValTok{100}
\NormalTok{abs.nac <-}\StringTok{ }\KeywordTok{svytotal}\NormalTok{(}\OperatorTok{~}\NormalTok{si }\OperatorTok{+}\StringTok{ }\NormalTok{no }\OperatorTok{+}\StringTok{ }\NormalTok{ne, dsg.envi)}
\NormalTok{total.nac <-}\StringTok{ }\KeywordTok{svytotal}\NormalTok{(}\OperatorTok{~}\NormalTok{total, dsg.envi)}

\CommentTok{# por entidadad }
\NormalTok{rel.ent <-}\StringTok{ }\KeywordTok{svyby}\NormalTok{(}\OperatorTok{~}\NormalTok{si }\OperatorTok{+}\NormalTok{no }\OperatorTok{+}\StringTok{ }\NormalTok{ne,}\OperatorTok{~}\NormalTok{ENT,}\DataTypeTok{design=}\NormalTok{dsg.envi, svymean)}
\NormalTok{abs.ent <-}\StringTok{ }\KeywordTok{svyby}\NormalTok{(}\OperatorTok{~}\NormalTok{si}\OperatorTok{+}\StringTok{ }\NormalTok{no }\OperatorTok{+}\StringTok{ }\NormalTok{ne,}\OperatorTok{~}\NormalTok{ENT,}\DataTypeTok{design=}\NormalTok{dsg.envi, svytotal)}
\NormalTok{total.ent <-}\StringTok{ }\KeywordTok{svyby}\NormalTok{(}\OperatorTok{~}\NormalTok{total, }\OperatorTok{~}\NormalTok{ENT,}\DataTypeTok{design=}\NormalTok{dsg.envi, svytotal)}

\CommentTok{#anexar los nombres de las entidades (para hacer las tablas)}
\NormalTok{entidades=}\KeywordTok{c}\NormalTok{(}\StringTok{"AGU"}\NormalTok{, }\StringTok{"BCN"}\NormalTok{, }\StringTok{"BCS"}\NormalTok{, }\StringTok{"CAM"}\NormalTok{, }\StringTok{"COA"}\NormalTok{, }\StringTok{"COL"}\NormalTok{, }\StringTok{"CHP"}\NormalTok{, }\StringTok{"CHH"}\NormalTok{, }\StringTok{"CMX"}\NormalTok{, }
            \StringTok{"DUR"}\NormalTok{, }\StringTok{"GUA"}\NormalTok{, }\StringTok{"GRO"}\NormalTok{, }\StringTok{"HID"}\NormalTok{, }\StringTok{"JAL"}\NormalTok{, }\StringTok{"MEX"}\NormalTok{, }\StringTok{"MIC"}\NormalTok{, }\StringTok{"MOR"}\NormalTok{, }\StringTok{"NAY"}\NormalTok{, }\StringTok{"NLE"}\NormalTok{,}
            \StringTok{"OAX"}\NormalTok{, }\StringTok{"PUE"}\NormalTok{, }\StringTok{"QUE"}\NormalTok{, }\StringTok{"ROO"}\NormalTok{, }\StringTok{"SLP"}\NormalTok{, }\StringTok{"SIN"}\NormalTok{, }\StringTok{"SON"}\NormalTok{, }\StringTok{"TAB"}\NormalTok{, }\StringTok{"TAM"}\NormalTok{, }\StringTok{"TLA"}\NormalTok{,}
            \StringTok{"VER"}\NormalTok{, }\StringTok{"YUC"}\NormalTok{, }\StringTok{"ZAC"}\NormalTok{)}

\CommentTok{#mostramos los resultados}
\end{Highlighting}
\end{Shaded}

\begin{Shaded}
\begin{Highlighting}[]
\NormalTok{total.nacdf <-}\StringTok{ }\KeywordTok{as.data.frame}\NormalTok{(total.nac)}
\KeywordTok{colnames}\NormalTok{(total.nacdf) <-}\StringTok{ }\KeywordTok{c}\NormalTok{(}\StringTok{"total"}\NormalTok{, }\StringTok{"se"}\NormalTok{)}
\KeywordTok{row.names}\NormalTok{(total.nacdf) <-}\StringTok{ }\KeywordTok{c}\NormalTok{(}\StringTok{"Nivel nacional"}\NormalTok{)}

\NormalTok{total.entdf <-}\StringTok{ }\NormalTok{total.ent[, }\KeywordTok{c}\NormalTok{(}\StringTok{"total"}\NormalTok{, }\StringTok{"se"}\NormalTok{)]}
\KeywordTok{row.names}\NormalTok{(total.entdf) <-}\StringTok{ }\NormalTok{entidades}

\NormalTok{dftotal <-}\StringTok{ }\KeywordTok{union_all}\NormalTok{(total.nacdf, total.entdf)}
\KeywordTok{colnames}\NormalTok{(dftotal) <-}\StringTok{ }\KeywordTok{c}\NormalTok{(}\StringTok{"Total"}\NormalTok{, }\StringTok{"Error estándar"}\NormalTok{)}

\KeywordTok{kbl}\NormalTok{(dftotal, }\DataTypeTok{caption =} \StringTok{"Total de hogares"}\NormalTok{, }\DataTypeTok{booktabs =}\NormalTok{ T) }\OperatorTok\StringTok{ }
\StringTok{  }\KeywordTok{kable_styling}\NormalTok{(}\DataTypeTok{latex_options =} \KeywordTok{c}\NormalTok{(}\StringTok{"striped"}\NormalTok{, }\StringTok{"HOLD_position"}\NormalTok{))}
\end{Highlighting}
\end{Shaded}

\begin{table}[H]

\caption{\label{tab:unnamed-chunk-13}Total de hogares}
\centering
\begin{tabular}[t]{lrr}
\toprule
  & Total & Error estándar\\
\midrule
\cellcolor{gray!6}{Nivel nacional} & \cellcolor{gray!6}{36210467} & \cellcolor{gray!6}{250260.723090}\\
AGU & 397770 & 9362.560429\\
\cellcolor{gray!6}{BCN} & \cellcolor{gray!6}{1156528} & \cellcolor{gray!6}{30667.900587}\\
BCS & 246920 & 9038.064657\\
\cellcolor{gray!6}{CAM} & \cellcolor{gray!6}{262489} & \cellcolor{gray!6}{10080.177465}\\
\addlinespace
COA & 913569 & 20581.753539\\
\cellcolor{gray!6}{COL} & \cellcolor{gray!6}{234272} & \cellcolor{gray!6}{8612.212124}\\
CHP & 1460368 & 65259.754453\\
\cellcolor{gray!6}{CHH} & \cellcolor{gray!6}{1147667} & \cellcolor{gray!6}{21729.674891}\\
CMX & 2808652 & 53774.155230\\
\addlinespace
\cellcolor{gray!6}{DUR} & \cellcolor{gray!6}{507158} & \cellcolor{gray!6}{12451.339215}\\
GUA & 1663749 & 43662.953028\\
\cellcolor{gray!6}{GRO} & \cellcolor{gray!6}{969487} & \cellcolor{gray!6}{29227.595625}\\
HID & 879538 & 25147.753568\\
\cellcolor{gray!6}{JAL} & \cellcolor{gray!6}{2384946} & \cellcolor{gray!6}{60219.557229}\\
\addlinespace
MEX & 4801185 & 154102.191785\\
\cellcolor{gray!6}{MIC} & \cellcolor{gray!6}{1340554} & \cellcolor{gray!6}{48980.807753}\\
MOR & 593961 & 20541.243433\\
\cellcolor{gray!6}{NAY} & \cellcolor{gray!6}{364784} & \cellcolor{gray!6}{10262.248527}\\
NLE & 1702725 & 47509.691132\\
\addlinespace
\cellcolor{gray!6}{OAX} & \cellcolor{gray!6}{1157915} & \cellcolor{gray!6}{53248.154463}\\
PUE & 1777565 & 61405.772431\\
\cellcolor{gray!6}{QUE} & \cellcolor{gray!6}{680255} & \cellcolor{gray!6}{33271.499208}\\
ROO & 563868 & 22050.121615\\
\cellcolor{gray!6}{SLP} & \cellcolor{gray!6}{790881} & \cellcolor{gray!6}{16199.544290}\\
\addlinespace
SIN & 870827 & 19436.070996\\
\cellcolor{gray!6}{SON} & \cellcolor{gray!6}{883105} & \cellcolor{gray!6}{26162.194677}\\
TAB & 694930 & 25398.594154\\
\cellcolor{gray!6}{TAM} & \cellcolor{gray!6}{1058100} & \cellcolor{gray!6}{22381.889965}\\
TLA & 347967 & 9438.489241\\
\addlinespace
\cellcolor{gray!6}{VER} & \cellcolor{gray!6}{2402304} & \cellcolor{gray!6}{75073.973302}\\
YUC & 683612 & 23033.342983\\
\cellcolor{gray!6}{ZAC} & \cellcolor{gray!6}{462816} & \cellcolor{gray!6}{16348.792929}\\
\bottomrule
\end{tabular}
\end{table}

\begin{Shaded}
\begin{Highlighting}[]
\NormalTok{abs.nacdf <-}\StringTok{ }\KeywordTok{as.data.frame}\NormalTok{(abs.nac)}
\KeywordTok{colnames}\NormalTok{(abs.nacdf) <-}\StringTok{ }\KeywordTok{c}\NormalTok{(}\StringTok{"Absoluto"}\NormalTok{, }\StringTok{"Error estándar de absoluto"}\NormalTok{)}
\NormalTok{rel.nacdf <-}\StringTok{ }\KeywordTok{as.data.frame}\NormalTok{(rel.nac)}
\KeywordTok{colnames}\NormalTok{(rel.nacdf) <-}\StringTok{ }\KeywordTok{c}\NormalTok{(}\StringTok{"Relativo"}\NormalTok{, }\StringTok{"Error estándar de relativo"}\NormalTok{)}


\NormalTok{n.nac <-}\StringTok{ }\KeywordTok{cbind}\NormalTok{(abs.nacdf, rel.nacdf)}
\KeywordTok{rownames}\NormalTok{(n.nac) <-}\StringTok{ }\KeywordTok{c}\NormalTok{(}\StringTok{"Si"}\NormalTok{, }\StringTok{"No"}\NormalTok{, }\StringTok{"No sabe"}\NormalTok{)}
  
\KeywordTok{kbl}\NormalTok{(n.nac, }
    \DataTypeTok{caption =} \StringTok{"Condición de hogares con necesidad de vivienda, nivel nacional"}\NormalTok{, }
    \DataTypeTok{booktabs =}\NormalTok{ T) }\OperatorTok\StringTok{ }\KeywordTok{kable_styling}\NormalTok{(}\DataTypeTok{latex_options =} \KeywordTok{c}\NormalTok{(}\StringTok{"striped"}\NormalTok{, }\StringTok{"HOLD_position"}\NormalTok{))}
\end{Highlighting}
\end{Shaded}

\begin{table}[H]

\caption{\label{tab:unnamed-chunk-14}Condición de hogares con necesidad de vivienda, nivel nacional}
\centering
\begin{tabular}[t]{lrrrr}
\toprule
  & Absoluto & Error estándar de absoluto & Relativo & Error estándar de relativo\\
\midrule
\cellcolor{gray!6}{Si} & \cellcolor{gray!6}{7628562} & \cellcolor{gray!6}{105958.232900} & \cellcolor{gray!6}{21.0672842192} & \cellcolor{gray!6}{0.0025763631}\\
No & 28529481 & 220719.483131 & 78.7879399622 & 0.0025834577\\
\cellcolor{gray!6}{No sabe} & \cellcolor{gray!6}{52424} & \cellcolor{gray!6}{7310.702217} & \cellcolor{gray!6}{0.1447758185} & \cellcolor{gray!6}{0.0002021010}\\
\bottomrule
\end{tabular}
\end{table}

SE significará Error estándar.

\begin{Shaded}
\begin{Highlighting}[]
\NormalTok{abs.entdf <-}\StringTok{ }\KeywordTok{as.data.frame}\NormalTok{(abs.ent)}
\NormalTok{abs.entdf <-}\StringTok{ }\NormalTok{abs.entdf[}\KeywordTok{c}\NormalTok{(}\OperatorTok{-}\DecValTok{1}\NormalTok{)]}
\KeywordTok{row.names}\NormalTok{(abs.entdf) <-}\StringTok{ }\NormalTok{entidades}
\KeywordTok{colnames}\NormalTok{(abs.entdf) <-}\StringTok{ }\KeywordTok{c}\NormalTok{(}\StringTok{"Si"}\NormalTok{, }\StringTok{"NO"}\NormalTok{, }\StringTok{"No sabe"}\NormalTok{, }\StringTok{"SE-Si"}\NormalTok{, }\StringTok{"SE-No"}\NormalTok{, }
                         \StringTok{"SE-No sabe"}\NormalTok{)}

\KeywordTok{kbl}\NormalTok{(abs.entdf, }
    \DataTypeTok{caption =} \StringTok{"Condición de hogares con necesidad de vivienda, absoluto nivel entidad"}\NormalTok{, }
    \DataTypeTok{booktabs =}\NormalTok{ T) }\OperatorTok\StringTok{ }\KeywordTok{kable_styling}\NormalTok{(}\DataTypeTok{latex_options =} \KeywordTok{c}\NormalTok{(}\StringTok{"striped"}\NormalTok{, }\StringTok{"HOLD_position"}\NormalTok{))}
\end{Highlighting}
\end{Shaded}

\begin{table}[H]

\caption{\label{tab:unnamed-chunk-15}Condición de hogares con necesidad de vivienda, absoluto nivel entidad}
\centering
\begin{tabular}[t]{lrrrrrr}
\toprule
  & Si & NO & No sabe & SE-Si & SE-No & SE-No sabe\\
\midrule
\cellcolor{gray!6}{AGU} & \cellcolor{gray!6}{61505} & \cellcolor{gray!6}{335194} & \cellcolor{gray!6}{1071} & \cellcolor{gray!6}{3974.823389} & \cellcolor{gray!6}{8564.766518} & \cellcolor{gray!6}{475.7383443}\\
BCN & 281959 & 872108 & 2461 & 19337.419329 & 27919.457494 & 1244.3162781\\
\cellcolor{gray!6}{BCS} & \cellcolor{gray!6}{64373} & \cellcolor{gray!6}{182547} & \cellcolor{gray!6}{0} & \cellcolor{gray!6}{5173.874742} & \cellcolor{gray!6}{6449.383018} & \cellcolor{gray!6}{0.0000000}\\
CAM & 67321 & 195002 & 166 & 4445.643176 & 8161.858495 & 129.0968629\\
\cellcolor{gray!6}{COA} & \cellcolor{gray!6}{163939} & \cellcolor{gray!6}{747788} & \cellcolor{gray!6}{1842} & \cellcolor{gray!6}{10105.564449} & \cellcolor{gray!6}{19628.361907} & \cellcolor{gray!6}{933.9657381}\\
\addlinespace
COL & 40317 & 193611 & 344 & 3021.593299 & 7590.486296 & 201.1989066\\
\cellcolor{gray!6}{CHP} & \cellcolor{gray!6}{400920} & \cellcolor{gray!6}{1057487} & \cellcolor{gray!6}{1961} & \cellcolor{gray!6}{24523.240800} & \cellcolor{gray!6}{51431.143341} & \cellcolor{gray!6}{1401.0014276}\\
CHH & 180625 & 965828 & 1214 & 11366.583223 & 22561.522406 & 882.9099614\\
\cellcolor{gray!6}{CMX} & \cellcolor{gray!6}{759121} & \cellcolor{gray!6}{2046170} & \cellcolor{gray!6}{3361} & \cellcolor{gray!6}{39095.924432} & \cellcolor{gray!6}{48555.506365} & \cellcolor{gray!6}{2378.1255223}\\
DUR & 98252 & 408397 & 509 & 5787.977125 & 12180.408076 & 359.9236030\\
\addlinespace
\cellcolor{gray!6}{GUA} & \cellcolor{gray!6}{329132} & \cellcolor{gray!6}{1330149} & \cellcolor{gray!6}{4468} & \cellcolor{gray!6}{18324.931722} & \cellcolor{gray!6}{40831.519700} & \cellcolor{gray!6}{2044.6735681}\\
GRO & 303266 & 665626 & 595 & 16281.614662 & 24941.543274 & 595.0000000\\
\cellcolor{gray!6}{HID} & \cellcolor{gray!6}{140927} & \cellcolor{gray!6}{737813} & \cellcolor{gray!6}{798} & \cellcolor{gray!6}{9331.489836} & \cellcolor{gray!6}{23459.519342} & \cellcolor{gray!6}{566.4379931}\\
JAL & 425646 & 1956569 & 2731 & 24713.322114 & 55730.575500 & 1931.9578153\\
\cellcolor{gray!6}{MEX} & \cellcolor{gray!6}{883449} & \cellcolor{gray!6}{3909988} & \cellcolor{gray!6}{7748} & \cellcolor{gray!6}{56765.649370} & \cellcolor{gray!6}{136796.664492} & \cellcolor{gray!6}{3909.3186087}\\
\addlinespace
MIC & 258117 & 1079968 & 2469 & 17019.088202 & 41651.859837 & 1348.1427966\\
\cellcolor{gray!6}{MOR} & \cellcolor{gray!6}{109093} & \cellcolor{gray!6}{484504} & \cellcolor{gray!6}{364} & \cellcolor{gray!6}{6113.353616} & \cellcolor{gray!6}{20808.299349} & \cellcolor{gray!6}{364.0000000}\\
NAY & 81396 & 283203 & 185 & 5316.975140 & 8445.249760 & 185.0000000\\
\cellcolor{gray!6}{NLE} & \cellcolor{gray!6}{197403} & \cellcolor{gray!6}{1503188} & \cellcolor{gray!6}{2134} & \cellcolor{gray!6}{17956.984197} & \cellcolor{gray!6}{45934.192957} & \cellcolor{gray!6}{1237.1960233}\\
OAX & 293325 & 864590 & 0 & 21077.694128 & 42503.529453 & 0.0000000\\
\addlinespace
\cellcolor{gray!6}{PUE} & \cellcolor{gray!6}{425198} & \cellcolor{gray!6}{1351558} & \cellcolor{gray!6}{809} & \cellcolor{gray!6}{28040.784324} & \cellcolor{gray!6}{54736.050440} & \cellcolor{gray!6}{809.0000000}\\
QUE & 141001 & 537897 & 1357 & 11145.868274 & 28477.477284 & 894.6390334\\
\cellcolor{gray!6}{ROO} & \cellcolor{gray!6}{138127} & \cellcolor{gray!6}{424299} & \cellcolor{gray!6}{1442} & \cellcolor{gray!6}{9283.650409} & \cellcolor{gray!6}{20497.536257} & \cellcolor{gray!6}{663.2073582}\\
SLP & 123674 & 666332 & 875 & 8709.998494 & 14608.143397 & 623.5006014\\
\cellcolor{gray!6}{SIN} & \cellcolor{gray!6}{222713} & \cellcolor{gray!6}{645528} & \cellcolor{gray!6}{2586} & \cellcolor{gray!6}{9220.053952} & \cellcolor{gray!6}{19008.975118} & \cellcolor{gray!6}{1185.5167650}\\
\addlinespace
SON & 222551 & 659580 & 974 & 10768.713563 & 22849.422869 & 689.8594060\\
\cellcolor{gray!6}{TAB} & \cellcolor{gray!6}{192553} & \cellcolor{gray!6}{501518} & \cellcolor{gray!6}{859} & \cellcolor{gray!6}{12898.737387} & \cellcolor{gray!6}{20890.774502} & \cellcolor{gray!6}{613.4011738}\\
TAM & 166002 & 890294 & 1804 & 12338.637054 & 18993.140649 & 1053.2302692\\
\cellcolor{gray!6}{TLA} & \cellcolor{gray!6}{94156} & \cellcolor{gray!6}{253211} & \cellcolor{gray!6}{600} & \cellcolor{gray!6}{5211.782240} & \cellcolor{gray!6}{8215.154640} & \cellcolor{gray!6}{346.0028188}\\
VER & 536212 & 1860529 & 5563 & 31913.895347 & 62997.583973 & 2940.6361557\\
\addlinespace
\cellcolor{gray!6}{YUC} & \cellcolor{gray!6}{162218} & \cellcolor{gray!6}{520260} & \cellcolor{gray!6}{1134} & \cellcolor{gray!6}{9319.293354} & \cellcolor{gray!6}{21188.653382} & \cellcolor{gray!6}{641.7367061}\\
ZAC & 64071 & 398745 & 0 & 4393.677183 & 15235.175061 & 0.0000000\\
\bottomrule
\end{tabular}
\end{table}

\begin{Shaded}
\begin{Highlighting}[]
\NormalTok{rel.entdf <-}\StringTok{ }\KeywordTok{as.data.frame}\NormalTok{(rel.ent)}
\NormalTok{rel.entdf <-}\StringTok{ }\NormalTok{rel.entdf[}\KeywordTok{c}\NormalTok{(}\OperatorTok{-}\DecValTok{1}\NormalTok{)]}
\NormalTok{rel.entdf <-}\StringTok{ }\NormalTok{rel.entdf}\OperatorTok{*}\DecValTok{100}
\KeywordTok{row.names}\NormalTok{(rel.entdf) <-}\StringTok{ }\NormalTok{entidades}
\KeywordTok{colnames}\NormalTok{(rel.entdf) <-}\StringTok{ }\KeywordTok{c}\NormalTok{(}\StringTok{"Si"}\NormalTok{, }\StringTok{"NO"}\NormalTok{, }\StringTok{"No sabe"}\NormalTok{, }\StringTok{"SE-Si"}\NormalTok{, }\StringTok{"SE-No"}\NormalTok{, }\StringTok{"SE-No sabe"}\NormalTok{)}

\KeywordTok{kbl}\NormalTok{(rel.entdf, }
    \DataTypeTok{caption =} \StringTok{"Condición de hogares con necesidad de vivienda, }
\StringTok{    relativo nivel entidad"}\NormalTok{, }
    \DataTypeTok{booktabs =}\NormalTok{ T) }\OperatorTok\StringTok{ }\KeywordTok{kable_styling}\NormalTok{(}\DataTypeTok{latex_options =} \KeywordTok{c}\NormalTok{(}\StringTok{"striped"}\NormalTok{, }\StringTok{"HOLD_position"}\NormalTok{))}
\end{Highlighting}
\end{Shaded}

\begin{table}[H]

\caption{\label{tab:unnamed-chunk-16}Condición de hogares con necesidad de vivienda, 
    relativo nivel entidad}
\centering
\begin{tabular}[t]{lrrrrrr}
\toprule
  & Si & NO & No sabe & SE-Si & SE-No & SE-No sabe\\
\midrule
\cellcolor{gray!6}{AGU} & \cellcolor{gray!6}{15.46245318} & \cellcolor{gray!6}{84.26829575} & \cellcolor{gray!6}{0.2692510747} & \cellcolor{gray!6}{0.9173243600} & \cellcolor{gray!6}{0.9131692986} & \cellcolor{gray!6}{0.1194644400}\\
BCN & 24.37978155 & 75.40742637 & 0.2127920811 & 1.4942211922 & 1.5010807182 & 0.1077664912\\
\cellcolor{gray!6}{BCS} & \cellcolor{gray!6}{26.07038717} & \cellcolor{gray!6}{73.92961283} & \cellcolor{gray!6}{0.0000000000} & \cellcolor{gray!6}{1.5627555969} & \cellcolor{gray!6}{1.5627555969} & \cellcolor{gray!6}{0.0000000000}\\
CAM & 25.64716998 & 74.28958928 & 0.0632407453 & 1.3411186351 & 1.3407851506 & 0.0493019537\\
\cellcolor{gray!6}{COA} & \cellcolor{gray!6}{17.94489524} & \cellcolor{gray!6}{81.85347795} & \cellcolor{gray!6}{0.2016268065} & \cellcolor{gray!6}{1.0390495679} & \cellcolor{gray!6}{1.0544507667} & \cellcolor{gray!6}{0.1023115714}\\
\addlinespace
COL & 17.20948299 & 82.64367914 & 0.1468378637 & 1.1218973803 & 1.1228392561 & 0.0860145133\\
\cellcolor{gray!6}{CHP} & \cellcolor{gray!6}{27.45335422} & \cellcolor{gray!6}{72.41236455} & \cellcolor{gray!6}{0.1342812223} & \cellcolor{gray!6}{1.2105917646} & \cellcolor{gray!6}{1.2190243948} & \cellcolor{gray!6}{0.0962142745}\\
CHH & 15.73845026 & 84.15576992 & 0.1057798124 & 0.9798720388 & 0.9821714014 & 0.0768822884\\
\cellcolor{gray!6}{CMX} & \cellcolor{gray!6}{27.02794793} & \cellcolor{gray!6}{72.85238613} & \cellcolor{gray!6}{0.1196659465} & \cellcolor{gray!6}{1.2233667823} & \cellcolor{gray!6}{1.2256745014} & \cellcolor{gray!6}{0.0847155908}\\
DUR & 19.37305534 & 80.52658146 & 0.1003632004 & 1.1069716028 & 1.1087668608 & 0.0710157920\\
\addlinespace
\cellcolor{gray!6}{GUA} & \cellcolor{gray!6}{19.78255133} & \cellcolor{gray!6}{79.94889854} & \cellcolor{gray!6}{0.2685501238} & \cellcolor{gray!6}{1.0433200798} & \cellcolor{gray!6}{1.0429584614} & \cellcolor{gray!6}{0.1219704690}\\
GRO & 31.28107958 & 68.65754775 & 0.0613726641 & 1.4375165230 & 1.4402629578 & 0.0610727222\\
\cellcolor{gray!6}{HID} & \cellcolor{gray!6}{16.02284381} & \cellcolor{gray!6}{83.88642674} & \cellcolor{gray!6}{0.0907294511} & \cellcolor{gray!6}{0.9930538975} & \cellcolor{gray!6}{0.9921498463} & \cellcolor{gray!6}{0.0644212040}\\
JAL & 17.84719654 & 82.03829353 & 0.1145099302 & 0.9626518326 & 0.9609447566 & 0.0808565847\\
\cellcolor{gray!6}{MEX} & \cellcolor{gray!6}{18.40064484} & \cellcolor{gray!6}{81.43797833} & \cellcolor{gray!6}{0.1613768268} & \cellcolor{gray!6}{1.0401216644} & \cellcolor{gray!6}{1.0441224162} & \cellcolor{gray!6}{0.0818720661}\\
\addlinespace
MIC & 19.25450224 & 80.56132017 & 0.1841775863 & 1.0366917379 & 1.0617585316 & 0.1007764780\\
\cellcolor{gray!6}{MOR} & \cellcolor{gray!6}{18.36703083} & \cellcolor{gray!6}{81.57168568} & \cellcolor{gray!6}{0.0612834849} & \cellcolor{gray!6}{1.1521323133} & \cellcolor{gray!6}{1.1520266693} & \cellcolor{gray!6}{0.0612800484}\\
NAY & 22.31347866 & 77.63580640 & 0.0507149436 & 1.2116094402 & 1.2183039255 & 0.0508143613\\
\cellcolor{gray!6}{NLE} & \cellcolor{gray!6}{11.59335771} & \cellcolor{gray!6}{88.28131378} & \cellcolor{gray!6}{0.1253285175} & \cellcolor{gray!6}{1.0141366495} & \cellcolor{gray!6}{1.0150154615} & \cellcolor{gray!6}{0.0727745498}\\
OAX & 25.33217032 & 74.66782968 & 0.0000000000 & 1.3738244479 & 1.3738244479 & 0.0000000000\\
\addlinespace
\cellcolor{gray!6}{PUE} & \cellcolor{gray!6}{23.92025046} & \cellcolor{gray!6}{76.03423785} & \cellcolor{gray!6}{0.0455116972} & \cellcolor{gray!6}{1.4095637862} & \cellcolor{gray!6}{1.4076675669} & \cellcolor{gray!6}{0.0456509261}\\
QUE & 20.72766830 & 79.07284768 & 0.1994840170 & 1.3677842207 & 1.3558211722 & 0.1303141533\\
\cellcolor{gray!6}{ROO} & \cellcolor{gray!6}{24.49633602} & \cellcolor{gray!6}{75.24793037} & \cellcolor{gray!6}{0.2557336114} & \cellcolor{gray!6}{1.5660673312} & \cellcolor{gray!6}{1.5736470468} & \cellcolor{gray!6}{0.1179642123}\\
SLP & 15.63749793 & 84.25186596 & 0.1106361134 & 1.0016746996 & 0.9989220344 & 0.0789018376\\
\cellcolor{gray!6}{SIN} & \cellcolor{gray!6}{25.57488456} & \cellcolor{gray!6}{74.12815634} & \cellcolor{gray!6}{0.2969590975} & \cellcolor{gray!6}{1.0507983930} & \cellcolor{gray!6}{1.0690384216} & \cellcolor{gray!6}{0.1358151994}\\
\addlinespace
SON & 25.20096704 & 74.68874030 & 0.1102926606 & 1.0695561376 & 1.0708116277 & 0.0782073361\\
\cellcolor{gray!6}{TAB} & \cellcolor{gray!6}{27.70825839} & \cellcolor{gray!6}{72.16813204} & \cellcolor{gray!6}{0.1236095722} & \cellcolor{gray!6}{1.5293479660} & \cellcolor{gray!6}{1.5338242974} & \cellcolor{gray!6}{0.0875441274}\\
TAM & 15.68868727 & 84.14081845 & 0.1704942822 & 1.0334478251 & 1.0310601940 & 0.0991233198\\
\cellcolor{gray!6}{TLA} & \cellcolor{gray!6}{27.05888777} & \cellcolor{gray!6}{72.76868209} & \cellcolor{gray!6}{0.1724301442} & \cellcolor{gray!6}{1.3022474001} & \cellcolor{gray!6}{1.2957662277} & \cellcolor{gray!6}{0.0995935673}\\
VER & 22.32073876 & 77.44769188 & 0.2315693601 & 1.1029239966 & 1.1054092821 & 0.1223369010\\
\addlinespace
\cellcolor{gray!6}{YUC} & \cellcolor{gray!6}{23.72954249} & \cellcolor{gray!6}{76.10457394} & \cellcolor{gray!6}{0.1658835714} & \cellcolor{gray!6}{1.2799885187} & \cellcolor{gray!6}{1.2794173714} & \cellcolor{gray!6}{0.0942916662}\\
ZAC & 13.84373055 & 86.15626945 & 0.0000000000 & 0.8878609643 & 0.8878609643 & 0.0000000000\\
\bottomrule
\end{tabular}
\end{table}

\begin{Shaded}
\begin{Highlighting}[]
\NormalTok{total.nac.ic <-}\StringTok{ }\KeywordTok{as.data.frame}\NormalTok{(}\KeywordTok{confint}\NormalTok{(total.nac)) }
\KeywordTok{rownames}\NormalTok{(total.nac.ic) <-}\StringTok{ "Nivel nacional"}
\NormalTok{total.ent.ic <-}\StringTok{ }\KeywordTok{as.data.frame}\NormalTok{(}\KeywordTok{confint}\NormalTok{(total.ent))}
\KeywordTok{rownames}\NormalTok{(total.ent.ic) <-}\StringTok{ }\NormalTok{entidades}

\NormalTok{total.ic <-}\StringTok{ }\KeywordTok{union_all}\NormalTok{(total.nac.ic, total.ent.ic)}

\KeywordTok{kbl}\NormalTok{(total.ic, }
    \DataTypeTok{caption =} \StringTok{"Intervalos confianza, total,}
\StringTok{    condición de hogares con necesidad de vivienda"}\NormalTok{, }
    \DataTypeTok{booktabs =}\NormalTok{ T) }\OperatorTok\StringTok{ }\KeywordTok{kable_styling}\NormalTok{(}\DataTypeTok{latex_options =} \KeywordTok{c}\NormalTok{(}\StringTok{"striped"}\NormalTok{, }\StringTok{"HOLD_position"}\NormalTok{))}
\end{Highlighting}
\end{Shaded}

\begin{table}[H]

\caption{\label{tab:unnamed-chunk-17}Intervalos confianza, total,
    condición de hogares con necesidad de vivienda}
\centering
\begin{tabular}[t]{lrr}
\toprule
  & 2.5 \% & 97.5 \%\\
\midrule
\cellcolor{gray!6}{Nivel nacional} & \cellcolor{gray!6}{35719964.9960} & \cellcolor{gray!6}{36700969.0040}\\
AGU & 379419.7188 & 416120.2812\\
\cellcolor{gray!6}{BCN} & \cellcolor{gray!6}{1096420.0194} & \cellcolor{gray!6}{1216635.9806}\\
BCS & 229205.7188 & 264634.2812\\
\cellcolor{gray!6}{CAM} & \cellcolor{gray!6}{242732.2152} & \cellcolor{gray!6}{282245.7848}\\
\addlinespace
COA & 873229.5043 & 953908.4957\\
\cellcolor{gray!6}{COL} & \cellcolor{gray!6}{217392.3744} & \cellcolor{gray!6}{251151.6256}\\
CHP & 1332461.2316 & 1588274.7684\\
\cellcolor{gray!6}{CHH} & \cellcolor{gray!6}{1105077.6198} & \cellcolor{gray!6}{1190256.3802}\\
CMX & 2703256.5924 & 2914047.4076\\
\addlinespace
\cellcolor{gray!6}{DUR} & \cellcolor{gray!6}{482753.8236} & \cellcolor{gray!6}{531562.1764}\\
GUA & 1578171.1846 & 1749326.8154\\
\cellcolor{gray!6}{GRO} & \cellcolor{gray!6}{912201.9652} & \cellcolor{gray!6}{1026772.0348}\\
HID & 830249.3087 & 928826.6913\\
\cellcolor{gray!6}{JAL} & \cellcolor{gray!6}{2266917.8367} & \cellcolor{gray!6}{2502974.1633}\\
\addlinespace
MEX & 4499150.2542 & 5103219.7458\\
\cellcolor{gray!6}{MIC} & \cellcolor{gray!6}{1244553.3809} & \cellcolor{gray!6}{1436554.6191}\\
MOR & 553700.9027 & 634221.0973\\
\cellcolor{gray!6}{NAY} & \cellcolor{gray!6}{344670.3625} & \cellcolor{gray!6}{384897.6375}\\
NLE & 1609607.7165 & 1795842.2835\\
\addlinespace
\cellcolor{gray!6}{OAX} & \cellcolor{gray!6}{1053550.5350} & \cellcolor{gray!6}{1262279.4650}\\
PUE & 1657211.8976 & 1897918.1024\\
\cellcolor{gray!6}{QUE} & \cellcolor{gray!6}{615044.0598} & \cellcolor{gray!6}{745465.9402}\\
ROO & 520650.5558 & 607085.4442\\
\cellcolor{gray!6}{SLP} & \cellcolor{gray!6}{759130.4766} & \cellcolor{gray!6}{822631.5234}\\
\addlinespace
SIN & 832733.0008 & 908920.9992\\
\cellcolor{gray!6}{SON} & \cellcolor{gray!6}{831828.0407} & \cellcolor{gray!6}{934381.9593}\\
TAB & 645149.6702 & 744710.3298\\
\cellcolor{gray!6}{TAM} & \cellcolor{gray!6}{1014232.3018} & \cellcolor{gray!6}{1101967.6982}\\
TLA & 329467.9010 & 366466.0990\\
\addlinespace
\cellcolor{gray!6}{VER} & \cellcolor{gray!6}{2255161.7162} & \cellcolor{gray!6}{2549446.2838}\\
YUC & 638467.4773 & 728756.5227\\
\cellcolor{gray!6}{ZAC} & \cellcolor{gray!6}{430772.9547} & \cellcolor{gray!6}{494859.0453}\\
\bottomrule
\end{tabular}
\end{table}

Ahora presentamos los intervalos de conianza, suponiendo que el
estimador sigue una distribución Normal.

\begin{Shaded}
\begin{Highlighting}[]
\NormalTok{abs.nac.ic <-}\StringTok{ }\KeywordTok{as.data.frame}\NormalTok{(}\KeywordTok{confint}\NormalTok{(abs.nac))}
\KeywordTok{rownames}\NormalTok{(abs.nac.ic) <-}\StringTok{ }\KeywordTok{c}\NormalTok{(}\StringTok{"Si"}\NormalTok{, }\StringTok{"No"}\NormalTok{, }\StringTok{"No sabe"}\NormalTok{)}

\KeywordTok{kbl}\NormalTok{(abs.nac.ic, }
    \DataTypeTok{caption =} \StringTok{"Intervalos confianza, absoluto nacional, }
\StringTok{    condición de hogares con necesidad de vivienda"}\NormalTok{, }
    \DataTypeTok{booktabs =}\NormalTok{ T) }\OperatorTok\StringTok{ }\KeywordTok{kable_styling}\NormalTok{(}\DataTypeTok{latex_options =} \KeywordTok{c}\NormalTok{(}\StringTok{"striped"}\NormalTok{, }\StringTok{"HOLD_position"}\NormalTok{))}
\end{Highlighting}
\end{Shaded}

\begin{table}[H]

\caption{\label{tab:unnamed-chunk-18}Intervalos confianza, absoluto nacional, 
    condición de hogares con necesidad de vivienda}
\centering
\begin{tabular}[t]{lrr}
\toprule
  & 2.5 \% & 97.5 \%\\
\midrule
\cellcolor{gray!6}{Si} & \cellcolor{gray!6}{7420887.67965} & \cellcolor{gray!6}{7836236.32035}\\
No & 28096878.76238 & 28962083.23762\\
\cellcolor{gray!6}{No sabe} & \cellcolor{gray!6}{38095.28695} & \cellcolor{gray!6}{66752.71305}\\
\bottomrule
\end{tabular}
\end{table}

\begin{Shaded}
\begin{Highlighting}[]
\NormalTok{rel.nac.ic <-}\StringTok{ }\KeywordTok{as.data.frame}\NormalTok{(}\KeywordTok{confint}\NormalTok{(rel.nac))}
\KeywordTok{rownames}\NormalTok{(rel.nac.ic) <-}\StringTok{ }\KeywordTok{c}\NormalTok{(}\StringTok{"Si"}\NormalTok{, }\StringTok{"No"}\NormalTok{, }\StringTok{"No sabe"}\NormalTok{)}

\KeywordTok{kbl}\NormalTok{(rel.nac.ic, }
    \DataTypeTok{caption =} \StringTok{"Intervalos confianza, relativo nacional, }
\StringTok{    condición de hogares con necesidad de vivienda"}\NormalTok{, }
    \DataTypeTok{booktabs =}\NormalTok{ T) }\OperatorTok\StringTok{ }\KeywordTok{kable_styling}\NormalTok{(}\DataTypeTok{latex_options =} \KeywordTok{c}\NormalTok{(}\StringTok{"striped"}\NormalTok{, }\StringTok{"HOLD_position"}\NormalTok{))}
\end{Highlighting}
\end{Shaded}

\begin{table}[H]

\caption{\label{tab:unnamed-chunk-19}Intervalos confianza, relativo nacional, 
    condición de hogares con necesidad de vivienda}
\centering
\begin{tabular}[t]{lrr}
\toprule
  & 2.5 \% & 97.5 \%\\
\midrule
\cellcolor{gray!6}{Si} & \cellcolor{gray!6}{21.0622346404} & \cellcolor{gray!6}{21.0723337980}\\
No & 78.7828764781 & 78.7930034463\\
\cellcolor{gray!6}{No sabe} & \cellcolor{gray!6}{0.1443797079} & \cellcolor{gray!6}{0.1451719292}\\
\bottomrule
\end{tabular}
\end{table}

\begin{Shaded}
\begin{Highlighting}[]
\NormalTok{abs.ent.ic <-}\StringTok{ }\KeywordTok{as.data.frame}\NormalTok{(}\KeywordTok{confint}\NormalTok{(abs.ent))}
\NormalTok{abs.ent.ic.si <-}\StringTok{ }\NormalTok{abs.ent.ic[}\DecValTok{1}\OperatorTok{:}\DecValTok{32}\NormalTok{, ]}
\KeywordTok{row.names}\NormalTok{(abs.ent.ic.si) <-}\StringTok{ }\NormalTok{entidades}
\KeywordTok{kbl}\NormalTok{(abs.ent.ic.si, }
    \DataTypeTok{caption =} \StringTok{"Intervalos confianza, Sí, absoluto entidad, }
\StringTok{    condición de hogares con necesidad de vivienda"}\NormalTok{, }
    \DataTypeTok{booktabs =}\NormalTok{ T) }\OperatorTok\StringTok{ }\KeywordTok{kable_styling}\NormalTok{(}\DataTypeTok{latex_options =} \KeywordTok{c}\NormalTok{(}\StringTok{"striped"}\NormalTok{, }\StringTok{"HOLD_position"}\NormalTok{))}
\end{Highlighting}
\end{Shaded}

\begin{table}[H]

\caption{\label{tab:unnamed-chunk-20}Intervalos confianza, Sí, absoluto entidad, 
    condición de hogares con necesidad de vivienda}
\centering
\begin{tabular}[t]{lrr}
\toprule
  & 2.5 \% & 97.5 \%\\
\midrule
\cellcolor{gray!6}{AGU} & \cellcolor{gray!6}{53714.48931} & \cellcolor{gray!6}{69295.51069}\\
BCN & 244058.35456 & 319859.64544\\
\cellcolor{gray!6}{BCS} & \cellcolor{gray!6}{54232.39184} & \cellcolor{gray!6}{74513.60816}\\
CAM & 58607.69949 & 76034.30051\\
\cellcolor{gray!6}{COA} & \cellcolor{gray!6}{144132.45764} & \cellcolor{gray!6}{183745.54236}\\
\addlinespace
COL & 34394.78596 & 46239.21404\\
\cellcolor{gray!6}{CHP} & \cellcolor{gray!6}{352855.33125} & \cellcolor{gray!6}{448984.66875}\\
CHH & 158346.90626 & 202903.09374\\
\cellcolor{gray!6}{CMX} & \cellcolor{gray!6}{682494.39617} & \cellcolor{gray!6}{835747.60383}\\
DUR & 86907.77329 & 109596.22671\\
\addlinespace
\cellcolor{gray!6}{GUA} & \cellcolor{gray!6}{293215.79381} & \cellcolor{gray!6}{365048.20619}\\
GRO & 271354.62165 & 335177.37835\\
\cellcolor{gray!6}{HID} & \cellcolor{gray!6}{122637.61600} & \cellcolor{gray!6}{159216.38400}\\
JAL & 377208.77872 & 474083.22128\\
\cellcolor{gray!6}{MEX} & \cellcolor{gray!6}{772190.37168} & \cellcolor{gray!6}{994707.62832}\\
\addlinespace
MIC & 224760.20007 & 291473.79993\\
\cellcolor{gray!6}{MOR} & \cellcolor{gray!6}{97111.04709} & \cellcolor{gray!6}{121074.95291}\\
NAY & 70974.92022 & 91817.07978\\
\cellcolor{gray!6}{NLE} & \cellcolor{gray!6}{162207.95770} & \cellcolor{gray!6}{232598.04230}\\
OAX & 252013.47863 & 334636.52137\\
\addlinespace
\cellcolor{gray!6}{PUE} & \cellcolor{gray!6}{370239.07263} & \cellcolor{gray!6}{480156.92737}\\
QUE & 119155.49961 & 162846.50039\\
\cellcolor{gray!6}{ROO} & \cellcolor{gray!6}{119931.37955} & \cellcolor{gray!6}{156322.62045}\\
SLP & 106602.71665 & 140745.28335\\
\cellcolor{gray!6}{SIN} & \cellcolor{gray!6}{204642.02632} & \cellcolor{gray!6}{240783.97368}\\
\addlinespace
SON & 201444.70926 & 243657.29074\\
\cellcolor{gray!6}{TAB} & \cellcolor{gray!6}{167271.93927} & \cellcolor{gray!6}{217834.06073}\\
TAM & 141818.71576 & 190185.28424\\
\cellcolor{gray!6}{TLA} & \cellcolor{gray!6}{83941.09452} & \cellcolor{gray!6}{104370.90548}\\
VER & 473661.91451 & 598762.08549\\
\addlinespace
\cellcolor{gray!6}{YUC} & \cellcolor{gray!6}{143952.52066} & \cellcolor{gray!6}{180483.47934}\\
ZAC & 55459.55096 & 72682.44904\\
\bottomrule
\end{tabular}
\end{table}

\begin{Shaded}
\begin{Highlighting}[]
\NormalTok{abs.ent.ic.no <-}\StringTok{ }\NormalTok{abs.ent.ic[}\DecValTok{33}\OperatorTok{:}\DecValTok{64}\NormalTok{, ]}
\KeywordTok{row.names}\NormalTok{(abs.ent.ic.no) <-}\StringTok{ }\NormalTok{entidades}
\KeywordTok{kbl}\NormalTok{(abs.ent.ic.no, }
    \DataTypeTok{caption =} \StringTok{"Intervalos confianza, No, absoluto entidad, }
\StringTok{    condición de hogares con necesidad de vivienda"}\NormalTok{, }
    \DataTypeTok{booktabs =}\NormalTok{ T) }\OperatorTok\StringTok{ }\KeywordTok{kable_styling}\NormalTok{(}\DataTypeTok{latex_options =} \KeywordTok{c}\NormalTok{(}\StringTok{"striped"}\NormalTok{, }\StringTok{"HOLD_position"}\NormalTok{))}
\end{Highlighting}
\end{Shaded}

\begin{table}[H]

\caption{\label{tab:unnamed-chunk-20}Intervalos confianza, No, absoluto entidad, 
    condición de hogares con necesidad de vivienda}
\centering
\begin{tabular}[t]{lrr}
\toprule
  & 2.5 \% & 97.5 \%\\
\midrule
\cellcolor{gray!6}{AGU} & \cellcolor{gray!6}{318407.3661} & \cellcolor{gray!6}{351980.6339}\\
BCN & 817386.8688 & 926829.1312\\
\cellcolor{gray!6}{BCS} & \cellcolor{gray!6}{169906.4416} & \cellcolor{gray!6}{195187.5584}\\
CAM & 179005.0513 & 210998.9487\\
\cellcolor{gray!6}{COA} & \cellcolor{gray!6}{709317.1176} & \cellcolor{gray!6}{786258.8824}\\
\addlinespace
COL & 178733.9202 & 208488.0798\\
\cellcolor{gray!6}{CHP} & \cellcolor{gray!6}{956683.8114} & \cellcolor{gray!6}{1158290.1886}\\
CHH & 921608.2286 & 1010047.7714\\
\cellcolor{gray!6}{CMX} & \cellcolor{gray!6}{1951002.9563} & \cellcolor{gray!6}{2141337.0437}\\
DUR & 384523.8389 & 432270.1611\\
\addlinespace
\cellcolor{gray!6}{GUA} & \cellcolor{gray!6}{1250120.6920} & \cellcolor{gray!6}{1410177.3080}\\
GRO & 616741.4735 & 714510.5265\\
\cellcolor{gray!6}{HID} & \cellcolor{gray!6}{691833.1870} & \cellcolor{gray!6}{783792.8130}\\
JAL & 1847339.0792 & 2065798.9208\\
\cellcolor{gray!6}{MEX} & \cellcolor{gray!6}{3641871.4644} & \cellcolor{gray!6}{4178104.5356}\\
\addlinespace
MIC & 998331.8548 & 1161604.1452\\
\cellcolor{gray!6}{MOR} & \cellcolor{gray!6}{443720.4827} & \cellcolor{gray!6}{525287.5173}\\
NAY & 266650.6146 & 299755.3854\\
\cellcolor{gray!6}{NLE} & \cellcolor{gray!6}{1413158.6361} & \cellcolor{gray!6}{1593217.3639}\\
OAX & 781284.6131 & 947895.3869\\
\addlinespace
\cellcolor{gray!6}{PUE} & \cellcolor{gray!6}{1244277.3125} & \cellcolor{gray!6}{1458838.6875}\\
QUE & 482082.1702 & 593711.8298\\
\cellcolor{gray!6}{ROO} & \cellcolor{gray!6}{384124.5672} & \cellcolor{gray!6}{464473.4328}\\
SLP & 637700.5651 & 694963.4349\\
\cellcolor{gray!6}{SIN} & \cellcolor{gray!6}{608271.0934} & \cellcolor{gray!6}{682784.9066}\\
\addlinespace
SON & 614795.9541 & 704364.0459\\
\cellcolor{gray!6}{TAB} & \cellcolor{gray!6}{460572.8344} & \cellcolor{gray!6}{542463.1656}\\
TAM & 853068.1284 & 927519.8716\\
\cellcolor{gray!6}{TLA} & \cellcolor{gray!6}{237109.5928} & \cellcolor{gray!6}{269312.4072}\\
VER & 1737056.0043 & 1984001.9957\\
\addlinespace
\cellcolor{gray!6}{YUC} & \cellcolor{gray!6}{478731.0025} & \cellcolor{gray!6}{561788.9975}\\
ZAC & 368884.6056 & 428605.3944\\
\bottomrule
\end{tabular}
\end{table}

\begin{Shaded}
\begin{Highlighting}[]
\NormalTok{abs.ent.ic.ne <-}\StringTok{ }\NormalTok{abs.ent.ic[}\DecValTok{65}\OperatorTok{:}\DecValTok{96}\NormalTok{, ]}
\KeywordTok{row.names}\NormalTok{(abs.ent.ic.ne) <-}\StringTok{ }\NormalTok{entidades}
\CommentTok{# esto es ya que no hay números negativos}
\NormalTok{abs.ent.ic.ne}\OperatorTok{$}\StringTok{`}\DataTypeTok{2.5 %}\StringTok{`}\NormalTok{ <-}\StringTok{ }\KeywordTok{pmax}\NormalTok{(}\DecValTok{0}\NormalTok{, abs.ent.ic.ne}\OperatorTok{$}\StringTok{`}\DataTypeTok{2.5 %}\StringTok{`}\NormalTok{)}
\KeywordTok{kbl}\NormalTok{(abs.ent.ic.ne, }
    \DataTypeTok{caption =} \StringTok{"Intervalos confianza, No sabe, absoluto entidad,}
\StringTok{    condición de hogares con necesidad de vivienda"}\NormalTok{, }
    \DataTypeTok{booktabs =}\NormalTok{ T) }\OperatorTok\StringTok{ }\KeywordTok{kable_styling}\NormalTok{(}\DataTypeTok{latex_options =} \KeywordTok{c}\NormalTok{(}\StringTok{"striped"}\NormalTok{, }\StringTok{"HOLD_position"}\NormalTok{))}
\end{Highlighting}
\end{Shaded}

\begin{table}[H]

\caption{\label{tab:unnamed-chunk-20}Intervalos confianza, No sabe, absoluto entidad,
    condición de hogares con necesidad de vivienda}
\centering
\begin{tabular}[t]{lrr}
\toprule
  & 2.5 \% & 97.5 \%\\
\midrule
\cellcolor{gray!6}{AGU} & \cellcolor{gray!6}{138.56997914} & \cellcolor{gray!6}{2003.4300209}\\
BCN & 22.18490950 & 4899.8150905\\
\cellcolor{gray!6}{BCS} & \cellcolor{gray!6}{0.00000000} & \cellcolor{gray!6}{0.0000000}\\
CAM & 0.00000000 & 419.0252017\\
\cellcolor{gray!6}{COA} & \cellcolor{gray!6}{11.46079047} & \cellcolor{gray!6}{3672.5392095}\\
\addlinespace
COL & 0.00000000 & 738.3426106\\
\cellcolor{gray!6}{CHP} & \cellcolor{gray!6}{0.00000000} & \cellcolor{gray!6}{4706.9123403}\\
CHH & 0.00000000 & 2944.4717260\\
\cellcolor{gray!6}{CMX} & \cellcolor{gray!6}{0.00000000} & \cellcolor{gray!6}{8022.0403745}\\
DUR & 0.00000000 & 1214.4372991\\
\addlinespace
\cellcolor{gray!6}{GUA} & \cellcolor{gray!6}{460.51344642} & \cellcolor{gray!6}{8475.4865536}\\
GRO & 0.00000000 & 1761.1785708\\
\cellcolor{gray!6}{HID} & \cellcolor{gray!6}{0.00000000} & \cellcolor{gray!6}{1908.1980659}\\
JAL & 0.00000000 & 6517.5677376\\
\cellcolor{gray!6}{MEX} & \cellcolor{gray!6}{85.87632281} & \cellcolor{gray!6}{15410.1236772}\\
\addlinespace
MIC & 0.00000000 & 5111.3113273\\
\cellcolor{gray!6}{MOR} & \cellcolor{gray!6}{0.00000000} & \cellcolor{gray!6}{1077.4268904}\\
NAY & 0.00000000 & 547.5933371\\
\cellcolor{gray!6}{NLE} & \cellcolor{gray!6}{0.00000000} & \cellcolor{gray!6}{4558.8596474}\\
OAX & 0.00000000 & 0.0000000\\
\addlinespace
\cellcolor{gray!6}{PUE} & \cellcolor{gray!6}{0.00000000} & \cellcolor{gray!6}{2394.6108635}\\
QUE & 0.00000000 & 3110.4602845\\
\cellcolor{gray!6}{ROO} & \cellcolor{gray!6}{142.13746360} & \cellcolor{gray!6}{2741.8625364}\\
SLP & 0.00000000 & 2097.0387232\\
\cellcolor{gray!6}{SIN} & \cellcolor{gray!6}{262.42983761} & \cellcolor{gray!6}{4909.5701624}\\
\addlinespace
SON & 0.00000000 & 2326.0995901\\
\cellcolor{gray!6}{TAB} & \cellcolor{gray!6}{0.00000000} & \cellcolor{gray!6}{2061.2442087}\\
TAM & 0.00000000 & 3868.2933951\\
\cellcolor{gray!6}{TLA} & \cellcolor{gray!6}{0.00000000} & \cellcolor{gray!6}{1278.1530634}\\
VER & 0.00000000 & 11326.5409567\\
\addlinespace
\cellcolor{gray!6}{YUC} & \cellcolor{gray!6}{0.00000000} & \cellcolor{gray!6}{2391.7808316}\\
ZAC & 0.00000000 & 0.0000000\\
\bottomrule
\end{tabular}
\end{table}

\begin{Shaded}
\begin{Highlighting}[]
\NormalTok{rel.ent.ic <-}\StringTok{ }\KeywordTok{as.data.frame}\NormalTok{(}\KeywordTok{confint}\NormalTok{(rel.ent)}\OperatorTok{*}\DecValTok{100}\NormalTok{)}
\NormalTok{rel.ent.ic.si <-}\StringTok{ }\NormalTok{rel.ent.ic[}\DecValTok{1}\OperatorTok{:}\DecValTok{32}\NormalTok{, ]}
\KeywordTok{row.names}\NormalTok{(rel.ent.ic.si) <-}\StringTok{ }\NormalTok{entidades}
\KeywordTok{kbl}\NormalTok{(rel.ent.ic.si, }
    \DataTypeTok{caption =} \StringTok{"Intervalos confianza, Sí, relativo entidad, }
\StringTok{    condición de hogares con necesidad de vivienda"}\NormalTok{, }
    \DataTypeTok{booktabs =}\NormalTok{ T) }\OperatorTok\StringTok{ }\KeywordTok{kable_styling}\NormalTok{(}\DataTypeTok{latex_options =} \KeywordTok{c}\NormalTok{(}\StringTok{"striped"}\NormalTok{, }\StringTok{"HOLD_position"}\NormalTok{))}
\end{Highlighting}
\end{Shaded}

\begin{table}[H]

\caption{\label{tab:unnamed-chunk-21}Intervalos confianza, Sí, relativo entidad, 
    condición de hogares con necesidad de vivienda}
\centering
\begin{tabular}[t]{lrr}
\toprule
  & 2.5 \% & 97.5 \%\\
\midrule
\cellcolor{gray!6}{AGU} & \cellcolor{gray!6}{13.664530469} & \cellcolor{gray!6}{17.26037588}\\
BCN & 21.451161831 & 27.30840127\\
\cellcolor{gray!6}{BCS} & \cellcolor{gray!6}{23.007442483} & \cellcolor{gray!6}{29.13333186}\\
CAM & 23.018625753 & 28.27571420\\
\cellcolor{gray!6}{COA} & \cellcolor{gray!6}{15.908395509} & \cellcolor{gray!6}{19.98139497}\\
\addlinespace
COL & 15.010604534 & 19.40836145\\
\cellcolor{gray!6}{CHP} & \cellcolor{gray!6}{25.080637964} & \cellcolor{gray!6}{29.82607048}\\
CHH & 13.817936359 & 17.65896417\\
\cellcolor{gray!6}{CMX} & \cellcolor{gray!6}{24.630193093} & \cellcolor{gray!6}{29.42570276}\\
DUR & 17.203430866 & 21.54267981\\
\addlinespace
\cellcolor{gray!6}{GUA} & \cellcolor{gray!6}{17.737681553} & \cellcolor{gray!6}{21.82742112}\\
GRO & 28.463598969 & 34.09856019\\
\cellcolor{gray!6}{HID} & \cellcolor{gray!6}{14.076493937} & \cellcolor{gray!6}{17.96919369}\\
JAL & 15.960433619 & 19.73395946\\
\cellcolor{gray!6}{MEX} & \cellcolor{gray!6}{16.362043839} & \cellcolor{gray!6}{20.43924584}\\
\addlinespace
MIC & 17.222623774 & 21.28638071\\
\cellcolor{gray!6}{MOR} & \cellcolor{gray!6}{16.108892992} & \cellcolor{gray!6}{20.62516867}\\
NAY & 19.938767795 & 24.68818953\\
\cellcolor{gray!6}{NLE} & \cellcolor{gray!6}{9.605686398} & \cellcolor{gray!6}{13.58102901}\\
OAX & 22.639523885 & 28.02481676\\
\addlinespace
\cellcolor{gray!6}{PUE} & \cellcolor{gray!6}{21.157556200} & \cellcolor{gray!6}{26.68294471}\\
QUE & 18.046860490 & 23.40847611\\
\cellcolor{gray!6}{ROO} & \cellcolor{gray!6}{21.426900455} & \cellcolor{gray!6}{27.56577159}\\
SLP & 13.674251594 & 17.60074426\\
\cellcolor{gray!6}{SIN} & \cellcolor{gray!6}{23.515357558} & \cellcolor{gray!6}{27.63441157}\\
\addlinespace
SON & 23.104675533 & 27.29725855\\
\cellcolor{gray!6}{TAB} & \cellcolor{gray!6}{24.710791453} & \cellcolor{gray!6}{30.70572532}\\
TAM & 13.663166753 & 17.71420779\\
\cellcolor{gray!6}{TLA} & \cellcolor{gray!6}{24.506529765} & \cellcolor{gray!6}{29.61124577}\\
VER & 20.159047446 & 24.48243007\\
\addlinespace
\cellcolor{gray!6}{YUC} & \cellcolor{gray!6}{21.220811092} & \cellcolor{gray!6}{26.23827389}\\
ZAC & 12.103555040 & 15.58390607\\
\bottomrule
\end{tabular}
\end{table}

\begin{Shaded}
\begin{Highlighting}[]
\NormalTok{rel.ent.ic.no <-}\StringTok{ }\NormalTok{rel.ent.ic[}\DecValTok{33}\OperatorTok{:}\DecValTok{64}\NormalTok{, ]}
\KeywordTok{row.names}\NormalTok{(rel.ent.ic.no) <-}\StringTok{ }\NormalTok{entidades}
\KeywordTok{kbl}\NormalTok{(rel.ent.ic.no, }
    \DataTypeTok{caption =} \StringTok{"Intervalos confianza, No, relativo entidad,}
\StringTok{    condición de hogares con necesidad de vivienda"}\NormalTok{, }
    \DataTypeTok{booktabs =}\NormalTok{ T) }\OperatorTok\StringTok{ }\KeywordTok{kable_styling}\NormalTok{(}\DataTypeTok{latex_options =} \KeywordTok{c}\NormalTok{(}\StringTok{"striped"}\NormalTok{, }\StringTok{"HOLD_position"}\NormalTok{))}
\end{Highlighting}
\end{Shaded}

\begin{table}[H]

\caption{\label{tab:unnamed-chunk-21}Intervalos confianza, No, relativo entidad,
    condición de hogares con necesidad de vivienda}
\centering
\begin{tabular}[t]{lrr}
\toprule
  & 2.5 \% & 97.5 \%\\
\midrule
\cellcolor{gray!6}{AGU} & \cellcolor{gray!6}{82.47851681} & \cellcolor{gray!6}{86.05807469}\\
BCN & 72.46536222 & 78.34949051\\
\cellcolor{gray!6}{BCS} & \cellcolor{gray!6}{70.86666814} & \cellcolor{gray!6}{76.99255752}\\
CAM & 71.66169867 & 76.91747988\\
\cellcolor{gray!6}{COA} & \cellcolor{gray!6}{79.78679243} & \cellcolor{gray!6}{83.92016348}\\
\addlinespace
COL & 80.44295464 & 84.84440364\\
\cellcolor{gray!6}{CHP} & \cellcolor{gray!6}{70.02312064} & \cellcolor{gray!6}{74.80160846}\\
CHH & 82.23074935 & 86.08079050\\
\cellcolor{gray!6}{CMX} & \cellcolor{gray!6}{70.45010825} & \cellcolor{gray!6}{75.25466401}\\
DUR & 78.35343835 & 82.69972457\\
\addlinespace
\cellcolor{gray!6}{GUA} & \cellcolor{gray!6}{77.90473752} & \cellcolor{gray!6}{81.99305956}\\
GRO & 65.83468423 & 71.48041128\\
\cellcolor{gray!6}{HID} & \cellcolor{gray!6}{81.94184877} & \cellcolor{gray!6}{85.83100470}\\
JAL & 80.15487642 & 83.92171064\\
\cellcolor{gray!6}{MEX} & \cellcolor{gray!6}{79.39153600} & \cellcolor{gray!6}{83.48442066}\\
\addlinespace
MIC & 78.48031169 & 82.64232865\\
\cellcolor{gray!6}{MOR} & \cellcolor{gray!6}{79.31375490} & \cellcolor{gray!6}{83.82961646}\\
NAY & 75.24797458 & 80.02363821\\
\cellcolor{gray!6}{NLE} & \cellcolor{gray!6}{86.29192003} & \cellcolor{gray!6}{90.27070752}\\
OAX & 71.97518324 & 77.36047612\\
\addlinespace
\cellcolor{gray!6}{PUE} & \cellcolor{gray!6}{73.27526011} & \cellcolor{gray!6}{78.79321558}\\
QUE & 76.41548701 & 81.73020835\\
\cellcolor{gray!6}{ROO} & \cellcolor{gray!6}{72.16363883} & \cellcolor{gray!6}{78.33222190}\\
SLP & 82.29401475 & 86.20971717\\
\cellcolor{gray!6}{SIN} & \cellcolor{gray!6}{72.03287953} & \cellcolor{gray!6}{76.22343314}\\
\addlinespace
SON & 72.58998807 & 76.78749252\\
\cellcolor{gray!6}{TAB} & \cellcolor{gray!6}{69.16189166} & \cellcolor{gray!6}{75.17437242}\\
TAM & 82.11997760 & 86.16165929\\
\cellcolor{gray!6}{TLA} & \cellcolor{gray!6}{70.22902695} & \cellcolor{gray!6}{75.30833723}\\
VER & 75.28112950 & 79.61425426\\
\addlinespace
\cellcolor{gray!6}{YUC} & \cellcolor{gray!6}{73.59696197} & \cellcolor{gray!6}{78.61218591}\\
ZAC & 84.41609393 & 87.89644496\\
\bottomrule
\end{tabular}
\end{table}

\begin{Shaded}
\begin{Highlighting}[]
\NormalTok{rel.ent.ic.ne <-}\StringTok{ }\NormalTok{rel.ent.ic[}\DecValTok{65}\OperatorTok{:}\DecValTok{96}\NormalTok{, ]}
\KeywordTok{row.names}\NormalTok{(rel.ent.ic.ne) <-}\StringTok{ }\NormalTok{entidades}
\CommentTok{# esto es ya que no hay números negativos}
\NormalTok{rel.ent.ic.ne}\OperatorTok{$}\StringTok{`}\DataTypeTok{2.5 %}\StringTok{`}\NormalTok{ <-}\StringTok{ }\KeywordTok{pmax}\NormalTok{(}\DecValTok{0}\NormalTok{, rel.ent.ic.ne}\OperatorTok{$}\StringTok{`}\DataTypeTok{2.5 %}\StringTok{`}\NormalTok{)}
\KeywordTok{kbl}\NormalTok{(rel.ent.ic.ne, }
    \DataTypeTok{caption =} \StringTok{"Intervalos confianza, No sabe, relativo entidad,}
\StringTok{    condición de hogares con necesidad de vivienda"}\NormalTok{, }
    \DataTypeTok{booktabs =}\NormalTok{ T) }\OperatorTok\StringTok{ }\KeywordTok{kable_styling}\NormalTok{(}\DataTypeTok{latex_options =} \KeywordTok{c}\NormalTok{(}\StringTok{"striped"}\NormalTok{, }\StringTok{"HOLD_position"}\NormalTok{))}
\end{Highlighting}
\end{Shaded}

\begin{table}[H]

\caption{\label{tab:unnamed-chunk-21}Intervalos confianza, No sabe, relativo entidad,
    condición de hogares con necesidad de vivienda}
\centering
\begin{tabular}[t]{lrr}
\toprule
  & 2.5 \% & 97.5 \%\\
\midrule
\cellcolor{gray!6}{AGU} & \cellcolor{gray!6}{0.0351050750} & \cellcolor{gray!6}{0.5033970745}\\
BCN & 0.0015736396 & 0.4240105227\\
\cellcolor{gray!6}{BCS} & \cellcolor{gray!6}{0.0000000000} & \cellcolor{gray!6}{0.0000000000}\\
CAM & 0.0000000000 & 0.1598707989\\
\cellcolor{gray!6}{COA} & \cellcolor{gray!6}{0.0010998113} & \cellcolor{gray!6}{0.4021538017}\\
\addlinespace
COL & 0.0000000000 & 0.3154232119\\
\cellcolor{gray!6}{CHP} & \cellcolor{gray!6}{0.0000000000} & \cellcolor{gray!6}{0.3228577350}\\
CHH & 0.0000000000 & 0.2564663288\\
\cellcolor{gray!6}{CMX} & \cellcolor{gray!6}{0.0000000000} & \cellcolor{gray!6}{0.2857054535}\\
DUR & 0.0000000000 & 0.2395515951\\
\addlinespace
\cellcolor{gray!6}{GUA} & \cellcolor{gray!6}{0.0294923973} & \cellcolor{gray!6}{0.5076078504}\\
GRO & 0.0000000000 & 0.1810730000\\
\cellcolor{gray!6}{HID} & \cellcolor{gray!6}{0.0000000000} & \cellcolor{gray!6}{0.2169926908}\\
JAL & 0.0000000000 & 0.2729859241\\
\cellcolor{gray!6}{MEX} & \cellcolor{gray!6}{0.0009105259} & \cellcolor{gray!6}{0.3218431276}\\
\addlinespace
MIC & 0.0000000000 & 0.3816958536\\
\cellcolor{gray!6}{MOR} & \cellcolor{gray!6}{0.0000000000} & \cellcolor{gray!6}{0.1813901728}\\
NAY & 0.0000000000 & 0.1503092617\\
\cellcolor{gray!6}{NLE} & \cellcolor{gray!6}{0.0000000000} & \cellcolor{gray!6}{0.2679640141}\\
OAX & 0.0000000000 & 0.0000000000\\
\addlinespace
\cellcolor{gray!6}{PUE} & \cellcolor{gray!6}{0.0000000000} & \cellcolor{gray!6}{0.1349858683}\\
QUE & 0.0000000000 & 0.4548950641\\
\cellcolor{gray!6}{ROO} & \cellcolor{gray!6}{0.0245280039} & \cellcolor{gray!6}{0.4869392190}\\
SLP & 0.0000000000 & 0.2652808734\\
\cellcolor{gray!6}{SIN} & \cellcolor{gray!6}{0.0307661982} & \cellcolor{gray!6}{0.5631519968}\\
\addlinespace
SON & 0.0000000000 & 0.2635762226\\
\cellcolor{gray!6}{TAB} & \cellcolor{gray!6}{0.0000000000} & \cellcolor{gray!6}{0.2951929090}\\
TAM & 0.0000000000 & 0.3647724190\\
\cellcolor{gray!6}{TLA} & \cellcolor{gray!6}{0.0000000000} & \cellcolor{gray!6}{0.3676299492}\\
VER & 0.0000000000 & 0.4713452800\\
\addlinespace
\cellcolor{gray!6}{YUC} & \cellcolor{gray!6}{0.0000000000} & \cellcolor{gray!6}{0.3506918412}\\
ZAC & 0.0000000000 & 0.0000000000\\
\bottomrule
\end{tabular}
\end{table}

\end{document}
